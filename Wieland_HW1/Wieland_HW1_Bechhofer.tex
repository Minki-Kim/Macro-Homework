\documentclass[11pt]{amsart}


\usepackage{geometry}                % See geometry.pdf to learn the layout options. There are lots.
\geometry{a4paper}                   % ... or a4paper or a5paper or ... 
%\geometry{landscape}                % Activate for for rotated page geometry
\usepackage[parfill]{parskip}    % Activate to begin paragraphs with an empty line rather than an indent
\usepackage{enumitem}
\usepackage{graphicx}
\usepackage{amssymb}
\usepackage{amsmath}
\usepackage{epstopdf}
\DeclareGraphicsRule{.tif}{png}{.png}{`convert #1 `dirname #1`/`basename #1 .tif`.png}


\title{Econ 220C Problem Set \# 1}
\author{Nathaniel Bechhofer}
%\date{}                                           % Activate to display a given date or no date

\begin{document}
	


	
\maketitle

\section{Questions from textbook}

\subsection{Romer 5.8}

\subsubsection*{(a)}

We can start with the full Lagrangian, after substituting

\[
C_t = K_t + Y_t - K_{t+1} =  K_t + A K_t + e_t - K_{t+1}  = (1 + A) K_t + e_t - K_{t+1}
\]
to get
\[
\mathcal{L} = E \left[ \sum_{t=0}^{\infty} \frac{u(C_t) + \lambda_t ((1 + A) K_t + e_t - K_{t+1})}{(1+\rho)^t} \right]
\]

which gives first order conditions with respect to $C_t$ and $K_{t+1}$ (which are chosen each period) of 
\[
u'(C_t) = \lambda_t
\]
and
\[
\lambda_t = \frac{(1+A) E[\lambda_{t+1}]}{(1+\rho)}
\]
and combining gives
\[
u'(C_t) = \frac{(1+A) E[u'(C_{t+1})]}{(1+\rho)}
\]
but since $A=\rho$, we are left with a more standard
\[
u'(C_t) = E[u'(C_{t+1})]
\]
and we can substitute for $u'(C_t)$ since we are given the form of the utility function to get 
\[
u'(C_t) = 1 - 2 \theta C_t
\]
and we now have 
\[
1 - 2 \theta C_t = E[1 - 2 \theta C_{t+1}]
\] 
and from linearity of expectation we can cancel terms to get
\[
C_t = E[C_{t+1}]
\]
as our Euler equation.

\subsubsection*{(b)}

Substituting the guessed form into the resource constraint
\[
C_t = (1 + A) K_t + e_t - K_{t+1}
\]
gets us
\[
\alpha + \beta K_t + \gamma e_t = (1 + A) K_t + e_t - K_{t+1}
\]
and upon rearranging we have
\[
K_{t+1} = (1 + A - \beta) K_t + (1-\gamma) e_t - \alpha
\]
as our function for future capital.

\subsubsection*{(c)}

We have given that 
\[
C_t = \alpha + \beta K_t + \gamma e_t 
\]
so we merely need to find 
\[
E[C_{t+1}] = E[\alpha + \beta K_{t+1} + \gamma e_{t+1}]
\]

By linearity and substituting our earlier result, we have
\[
E[C_{t+1}] = \alpha + \beta ((1 + A - \beta) K_t + (1-\gamma) e_t - \alpha) + \gamma E[e_{t+1}]
\]
Since we are given that $\varepsilon_t$ has expectation zero, we can substitute
\[
E[e_{t+1}] = \phi e_t
\]
(since we are merely applying the law of motion for $e$ in period $t+1$ rather than period $t$) and so we have
\[
E[C_{t+1}] = \alpha + \beta ((1 + A - \beta) K_t + (1-\gamma) e_t - \alpha) + \gamma \phi e_t
\]
which we can simplify as
\[
E[C_{t+1}] = (1 - \beta) \alpha + \beta (1 + A - \beta) K_t + (\beta - \gamma \beta + \gamma \phi) e_t
\]
and from the Euler equation we have
\[
\alpha + \beta K_t + \gamma e_t  = (1 - \beta) \alpha + \beta (1 + A - \beta) K_t + (\beta - \gamma \beta + \gamma \phi) e_t
\]
and so we will have
\begin{align*}
\alpha &= (1 - \beta) \alpha \\
\beta &= \beta (1 + A - \beta) \\
\gamma &= (\beta - \gamma \beta + \gamma \phi)
\end{align*}
which upon solving this system (with 3 equations and 3 unknowns since $A$ and $\phi$ are given) yield
\begin{align*}
\alpha &= 0 \\
\beta &= A\\
\gamma &= \frac{A}{1 + A - \phi}
\end{align*}
for the parameters.

\subsubsection*{(d)}

\subsection{Romer 5.9}

\subsubsection*{(a)}

We can start with the full Lagrangian, after substituting

\[
C_t = K_t + Y_t - K_{t+1} =  K_t + A K_t + - K_{t+1}  = (1 + A) K_t - K_{t+1}
\]
to get
\[
\mathcal{L} = E \left[ \sum_{t=0}^{\infty} \frac{u(C_t) + \lambda_t ((1 + A) K_t - K_{t+1})}{(1+\rho)^t} \right]
\]

which gives first order conditions with respect to $C_t$ and $K_{t+1}$ (which are chosen each period) of 
\[
u'(C_t) = \lambda_t
\]
and
\[
\lambda_t = \frac{(1+A) E[\lambda_{t+1}]}{(1+\rho)}
\]
and combining gives
\[
u'(C_t) = \frac{(1+A) E[u'(C_{t+1})]}{(1+\rho)}
\]
but since $A=\rho$, we are left with a more standard
\[
u'(C_t) = E[u'(C_{t+1})]
\]
and we can substitute for $u'(C_t)$ since we are given the form of the utility function to get 
\[
1 - 2 \theta (C_t + v_t) = 1 - 2 \theta (E[C_{t+1}] + E[v_t]) \implies C_t + v_t = E[C_{t+1}] 
\]
from linearity of expectation and the zero mean shock.

\subsubsection*{(b)}

Substituting the guessed form into the resource constraint
\[
C_t = (1 + A) K_t - K_{t+1}
\]
gets us
\[
\alpha + \beta K_t + \gamma v_t = (1 + A) K_t - K_{t+1}
\]
and upon rearranging we have
\[
K_{t+1} = (1 + A - \beta) K_t - \gamma v_t - \alpha
\]
as our function for future capital.

\subsubsection*{(c)}

We have given that 
\[
C_t = \alpha + \beta K_t + \gamma v_t 
\]
so we merely need to find 
\[
E[C_{t+1}] = E[\alpha + \beta K_{t+1} + \gamma v_{t+1}]
\]

By linearity and substituting our earlier result, we have
\[
E[C_{t+1}] = \alpha + \beta ((1 + A - \beta) K_t - \gamma v_t - \alpha) + \gamma E[v_{t+1}]
\]
Since we are given that $v_t$ has expectation zero, we can substitute to obtain
\[
E[C_{t+1}] = \alpha + \beta ((1 + A - \beta) K_t - \gamma v_t - \alpha)
\]
which we can simplify as
\[
E[C_{t+1}] = (1 - \beta) \alpha + \beta (1 + A - \beta) K_t - \gamma \beta v_t
\]
and from the Euler equation we have
\[
\alpha + \beta K_t + (\gamma + 1) v_t  = (1 - \beta) \alpha + \beta (1 + A - \beta) K_t - \gamma \beta v_t
\]
and so we will have
\begin{align*}
\alpha &= (1 - \beta) \alpha \\
\beta &= \beta (1 + A - \beta) \\
\gamma + 1 &= \gamma \beta
\end{align*}
which upon solving this system (with 3 equations and 3 unknowns since $A$ is given) yield 
\begin{align*}
\alpha &= 0 \\
\beta &= A\\
\gamma &= -\frac{1}{1 + A}
\end{align*}
for the parameters.

\subsubsection*{(d)}

Solving for consumption gives
\[
C_t = AK_t - \frac{1}{1 + A} \times v_t
\]
while saving is
\[
K_{t+1} = K_t + \frac{1}{1 + A} \times v_t
\]
and so a one time positive shock means that the capital stock is higher forever (since there is persistence), and this higher capital stock will mean both output and consumption are higher.


\subsection{Romer 5.11}

\subsubsection*{(a)}

This is the same for every value function: the choice of saving and investment must guarantee that utility in the current period is equal to the discounted expected value of the value function tomorrow, as otherwise the marginal value of allocating resources to today's utility would not be equal to the value of allocating resources for the future, violating the intertemporal first order conditions.


\section{Permanent income hypothesis and the ``excess smoothness'' puzzle}

\subsection{Saving responses to shocks}

\section{Estimation of adjustment costs}

\section{Practice log-linearization}

\subsection{}

\subsection{}

\subsection{}

\subsection{}

\subsection{}

\subsection{}

\subsection{}




\end{document}  