\documentclass[12pt]{article}


\usepackage{geometry}                % See geometry.pdf to learn the layout options. There are lots.
\geometry{a4paper}                   % ... or a4paper or a5paper or ...
%\geometry{landscape}                % Activate for for rotated page geometry
\usepackage[parfill]{parskip}    % Activate to begin paragraphs with an empty line rather than an indent
\usepackage{enumitem}
\usepackage{graphicx}
\usepackage{amssymb}
\usepackage{amsmath}
\usepackage{cancel}
\usepackage{epstopdf}
\DeclareGraphicsRule{.tif}{png}{.png}{`convert #1 `dirname #1`/`basename #1 .tif`.png}
\usepackage{breqn}

\title{Econ 220C Problem Set 1}
\author{Churn Ken Lee}
%\date{}                                           % Activate to display a given date or no date

\begin{document}




\maketitle

\section{Questions from textbook}

\subsection{Romer 5.8}

\subsubsection*{(a)}

We can start with the full Lagrangian, after substituting

\[
C_t = K_t + Y_t - K_{t+1} =  K_t + A K_t + e_t - K_{t+1}  = (1 + A) K_t + e_t - K_{t+1}
\]
to get
\[
\mathcal{L} = E \left[ \sum_{t=0}^{\infty} \frac{u(C_t) + \lambda_t ((1 + A) K_t + e_t - K_{t+1})}{(1+\rho)^t} \right]
\]

which gives first order conditions with respect to $C_t$ and $K_{t+1}$ (which are chosen each period) of
\[
u'(C_t) = \lambda_t
\]
and
\[
\lambda_t = \frac{(1+A) E[\lambda_{t+1}]}{(1+\rho)}
\]
and combining gives
\[
u'(C_t) = \frac{(1+A) E[u'(C_{t+1})]}{(1+\rho)}
\]
but since $A=\rho$, we are left with a more standard
\[
u'(C_t) = E[u'(C_{t+1})]
\]
and we can substitute for $u'(C_t)$ since we are given the form of the utility function to get
\[
u'(C_t) = 1 - 2 \theta C_t
\]
and we now have
\[
1 - 2 \theta C_t = E[1 - 2 \theta C_{t+1}]
\]
and from linearity of expectation we can cancel terms to get
\[
C_t = E[C_{t+1}]
\]
as our Euler equation.

\subsubsection*{(b)}

Substituting the guessed form into the resource constraint
\[
C_t = (1 + A) K_t + e_t - K_{t+1}
\]
gets us
\[
\alpha + \beta K_t + \gamma e_t = (1 + A) K_t + e_t - K_{t+1}
\]
and upon rearranging we have
\[
K_{t+1} = (1 + A - \beta) K_t + (1-\gamma) e_t - \alpha
\]
as our function for future capital.

\subsubsection*{(c)}

We have given that
\[
C_t = \alpha + \beta K_t + \gamma e_t
\]
so we merely need to find
\[
E[C_{t+1}] = E[\alpha + \beta K_{t+1} + \gamma e_{t+1}]
\]

By linearity and substituting our earlier result, we have
\[
E[C_{t+1}] = \alpha + \beta ((1 + A - \beta) K_t + (1-\gamma) e_t - \alpha) + \gamma E[e_{t+1}]
\]
Since we are given that $\varepsilon_t$ has expectation zero, we can substitute
\[
E[e_{t+1}] = \phi e_t
\]
(since we are merely applying the law of motion for $e$ in period $t+1$ rather than period $t$) and so we have
\[
E[C_{t+1}] = \alpha + \beta ((1 + A - \beta) K_t + (1-\gamma) e_t - \alpha) + \gamma \phi e_t
\]
which we can simplify as
\[
E[C_{t+1}] = (1 - \beta) \alpha + \beta (1 + A - \beta) K_t + (\beta - \gamma \beta + \gamma \phi) e_t
\]
and from the Euler equation we have
\[
\alpha + \beta K_t + \gamma e_t  = (1 - \beta) \alpha + \beta (1 + A - \beta) K_t + (\beta - \gamma \beta + \gamma \phi) e_t
\]
and so we will have
\begin{align*}
\alpha &= (1 - \beta) \alpha \\
\beta &= \beta (1 + A - \beta) \\
\gamma &= (\beta - \gamma \beta + \gamma \phi)
\end{align*}
which upon solving this system (with 3 equations and 3 unknowns since $A$ and $\phi$ are given) yield
\begin{align*}
\alpha &= 0 \\
\beta &= A\\
\gamma &= \frac{A}{1 + A - \phi}
\end{align*}
for the parameters.

\subsubsection*{(d)}

First we consider the case of positive $\phi$. Consumption rises and stays persistently higher, while capital increases as it approaches a new long-run equilibrium. Output jumps at first, but then declines as it approaches a new long-run level which is still higher than it was.

In the case of negative $\phi$, capital and output (and therefore consumption) will also eventually be higher, but they will oscillate around the new long-run level, as $\phi^s$ will be positive and negative depending on the period.

\subsection{Romer 5.9}

\subsubsection*{(a)}

We can start with the full Lagrangian, after substituting

\[
C_t = K_t + Y_t - K_{t+1} =  K_t + A K_t + - K_{t+1}  = (1 + A) K_t - K_{t+1}
\]
to get
\[
\mathcal{L} = E \left[ \sum_{t=0}^{\infty} \frac{u(C_t) + \lambda_t ((1 + A) K_t - K_{t+1})}{(1+\rho)^t} \right]
\]

which gives first order conditions with respect to $C_t$ and $K_{t+1}$ (which are chosen each period) of
\[
u'(C_t) = \lambda_t
\]
and
\[
\lambda_t = \frac{(1+A) E[\lambda_{t+1}]}{(1+\rho)}
\]
and combining gives
\[
u'(C_t) = \frac{(1+A) E[u'(C_{t+1})]}{(1+\rho)}
\]
but since $A=\rho$, we are left with a more standard
\[
u'(C_t) = E[u'(C_{t+1})]
\]
and we can substitute for $u'(C_t)$ since we are given the form of the utility function to get
\[
1 - 2 \theta (C_t + v_t) = 1 - 2 \theta (E[C_{t+1}] + E[v_{t+1}]) \implies C_t + v_t = E[C_{t+1}]
\]
from linearity of expectation and the zero mean shock.

\subsubsection*{(b)}

Substituting the guessed form into the resource constraint
\[
C_t = (1 + A) K_t - K_{t+1}
\]
gets us
\[
\alpha + \beta K_t + \gamma v_t = (1 + A) K_t - K_{t+1}
\]
and upon rearranging we have
\[
K_{t+1} = (1 + A - \beta) K_t - \gamma v_t - \alpha
\]
as our function for future capital.

\subsubsection*{(c)}

We have given that
\[
C_t = \alpha + \beta K_t + \gamma v_t
\]
so we merely need to find
\[
E[C_{t+1}] = E[\alpha + \beta K_{t+1} + \gamma v_{t+1}]
\]

By linearity and substituting our earlier result, we have
\[
E[C_{t+1}] = \alpha + \beta ((1 + A - \beta) K_t - \gamma v_t - \alpha) + \gamma E[v_{t+1}]
\]
Since we are given that $v_t$ has expectation zero, we can substitute to obtain
\[
E[C_{t+1}] = \alpha + \beta ((1 + A - \beta) K_t - \gamma v_t - \alpha)
\]
which we can simplify as
\[
E[C_{t+1}] = (1 - \beta) \alpha + \beta (1 + A - \beta) K_t - \gamma \beta v_t
\]
and from the Euler equation we have
\[
\alpha + \beta K_t + (\gamma + 1) v_t  = (1 - \beta) \alpha + \beta (1 + A - \beta) K_t - \gamma \beta v_t
\]
and so we will have
\begin{align*}
\alpha &= (1 - \beta) \alpha \\
\beta &= \beta (1 + A - \beta) \\
\gamma + 1 &= -\gamma \beta
\end{align*}
which upon solving this system (with 3 equations and 3 unknowns since $A$ is given) yield
\begin{align*}
\alpha &= 0 \\
\beta &= A\\
\gamma &= -\frac{1}{1 + A}
\end{align*}
for the parameters.

\subsubsection*{(d)}

Solving for consumption gives
\[
C_t = AK_t - \frac{1}{1 + A} \times v_t
\]
while saving is
\[
K_{t+1} = K_t + \frac{1}{1 + A} \times v_t
\]
and so a one time positive shock means that the capital stock is higher forever (since there is persistence), and this higher capital stock will mean both output and consumption are higher.


\subsection{Romer 5.11}

\subsubsection*{(a)}

This is the same for every value function: the choice of saving and investment must guarantee that utility in the current period is equal to the discounted expected value of the value function tomorrow, as otherwise the marginal value of allocating resources to today's utility would not be equal to the value of allocating resources for the future, violating the intertemporal first order conditions.

\subsubsection*{(b)}

The first order condition for consumption is
\[
\frac{1}{C_t} = -\frac{e^{-\rho} \beta_K}{Y_t - C_t}
\]
which we can rewrite as
\[
\frac{Y_t - C_t}{C_t} = -e^{-\rho} \beta_K
\]
and solving for $Y$ we have
\[
Y = C_t (1+ e^{-\rho} \beta_K)
\]
so consumption is a constant fraction of output.

\subsubsection*{(c)}

The first order condition for labor is
\[
-\frac{b}{1-L_t} = -\frac{e^{-\rho} \beta_K}{Y_t - C_t} \times (1-\alpha) K_t^{\alpha} A_t^{1-\alpha} L_t^{-\alpha}
\]
which we can rewrite using the definition of $Y$ and the first order condition for consumption as
\[
b L_t = (1-L_t) (1-\alpha) (1+ e^{-\rho} \beta_K)
\]
and so we have
\[
(1-L_t) = \frac{b}{(1-\alpha)(1+ e^{-\rho} \beta_K) + b}
\]
which shows that the labor supply is constant.

\subsubsection*{(d)}

Substitutions give
\begin{dmath*}
V(K_t, A_t) = \alpha \log K_t + (1-\alpha) \log A_t  + (1-\alpha) \log L_t - \log (1+ e^{-\rho} \beta_K) + b \log \left( \frac{b}{(1-\alpha)(1+ e^{-\rho} \beta_K) + b} \right) + e^{-\rho} \beta_0 + e^{-\rho} \beta_K [ \log e^{-\rho} \beta_k - \log (1+ e^{-\rho} \beta_K) + \alpha \log K_t + (1-\alpha) \log A_t  + (1-\alpha) \log L_t ] + e^{-\rho} \beta_A \rho_A \log A_t
\end{dmath*}
so we have
\[
\beta_K' = \alpha (1+ e^{-\rho} \beta_K)
\]
along with
\[
\beta_A' = (1-\alpha) (1+ e^{-\rho} \beta_K) + e^{-\rho} \beta_A \rho_A
\]
and
\begin{dmath*}
\beta_0' = (1-\alpha) \log L_t^* - \log (1+ e^{-\rho} \beta_K) + b \log \left( \frac{b}{(1-\alpha)(1+ e^{-\rho} \beta_K) + b} \right) + e^{-\rho} \beta_K \left[ \log e^{-\rho} \beta_k - \log (1+ e^{-\rho} \beta_K) + (1-\alpha) \log L_t^* \right]
\end{dmath*}

\subsubsection*{(e)}

We solve the system of equations
\begin{align*}
\beta_K &= \alpha (1+ e^{-\rho} \beta_K) \\
\beta_A &= (1-\alpha) (1+ e^{-\rho} \beta_K) + e^{-\rho} \beta_A \rho_A
\end{align*}

and so we get that
\[
\beta_K = \frac{\alpha}{1-\alpha e^{-\rho}}
\]
and
\[
\beta_A = \frac{1-\alpha}{(1-\alpha e^{-\rho})(1-\rho_A e^{-\rho})}
\]

\subsubsection*{(f)}

With this value of $\beta_K$, we get
\[
\frac{C_t}{Y_t} = \frac{1}{1 + \frac{\alpha e^{-\rho}}{1-\alpha e^{-\rho}}}
\]
which we can simplify as
\[
\frac{C_t}{Y_t} = 1-\alpha e^{-\rho}
\]
which is the same value.

For labor supply, we have
\[
L_t = \frac{1-\alpha}{(1-\alpha) + \frac{b}{1+ \frac{\alpha e^{-\rho}}{1-\alpha e^{-\rho}}}}
\]
and simplifying we have
\[
L_t = \frac{1-\alpha}{(1-\alpha) + b (1-\alpha e^{-\rho})}
\]
which is also the same value as in the earlier derivation.

\section{Permanent income hypothesis and the ``excess smoothness'' puzzle}

\subsection{Saving responses to shocks}
The Lagrangian is
\begin{align*}
	\mathcal{L} = E_t \left\lbrace \sum_{s=0}^\infty \beta^s U(C_{t+s}) + \lambda \left[ \sum_{k=0}^\infty \left( A_t - (1+r)^{-s} (C_{t+k} - Y_{t+k} ) \right) \right]\right\rbrace
\end{align*}
The first order conditions for $C_t$ and $C_{t+s}$ are respectively
\begin{align*}
	&U'(C_t) - \lambda = 0 \\
	&E_t \left[ \beta^s U'(C_{t+s}) -\lambda (1+r)^{-s} \right] = 0
\end{align*}
Given we have $\beta = (1+r)^{-1}$, we can combine the FOCs and we get
\[
U'(C_t) = E_t \left[ U'(C_{t+s}) \right]
\]
As utility is quadratic, this implies
\[
C_t = E_t \left[ C_{t+s} \right]
\]
Now we use the fact that the budget constraint holds in expectation,
\[
A_t = E_t \left[ \sum_{s=0}^\infty (1+r)^{-s} (C_{t+s} - T_{t+s}) \right]
\]
and we plug in $C_t = E_t \left[ C_{t+s} \right]$ to get
\[
C_t = (1-\beta) A_t + (1-\beta) \sum_{s=0}^\infty \beta^s E_t [Y_{t+s}]
\]

\subsubsection*{(a) $Y_t = \mu t + \phi Y_{t-1} + \epsilon_t$}
Assuming $\epsilon_k = 0$ for $k \neq t$, we can iterate the expression for $Y_{t+s}$ backwards to obtain
\[
Y_{t+s} = \sum_{k=0}^s \left[ \phi^k \mu (t+s-k) \right] + \phi^{s+1} Y_{t-1} + \phi^s \epsilon_t
\]
Hence
\[
\frac{\partial}{\partial \epsilon_t} E_t [Y_{t+s}] = \phi^s
\]
We can then use this to obtain
\begin{align*}
	\frac{\partial}{\partial \epsilon_t} C_t &= (1-\beta) \sum_{s=0}^\infty \beta^s \frac{\partial}{\partial \epsilon_t} E_t [Y_{t+s}] \\
	&= (1-\beta) \sum_{s=0}^\infty \beta^s \phi^s \\
	&= (1-\beta) \frac{1}{1-\beta \phi}
\end{align*}
Recall that savings is defined as $Y_{t+s} - C_{t+s}$, so the change in savings is
\[
\frac{\partial}{\partial \epsilon_t} E_t [Y_{t+s} - C_{t+s}] = \phi^s - (1-\beta) \frac{1}{1-\beta \phi} > 0
\]
Note that savings increase a lot initially but decreases over time. This is to be expected: given that the effect of an income shock diminishes over time (as $\phi \in (0,1)$), the increase in income is temporary and the consumer, who wants to smooth consumption, will want to save that increased initial income and spread the increased consumption across all periods.

\subsubsection*{(b) $Y_t = Y_{t-1} + \epsilon_t$}
Assuming $\epsilon_k = 0$ for $k \neq t$, we can iterate the expression for $Y_{t+s}$ backwards to obtain
\[
Y_{t+s} = Y_t + \epsilon_t
\]
Hence
\[
\frac{\partial}{\partial \epsilon_t} E_t [Y_{t+s}] = 1
\]
We can then use this to obtain
\begin{align*}
	\frac{\partial}{\partial \epsilon_t} C_t &= (1-\beta) \sum_{s=0}^\infty \beta^s \frac{\partial}{\partial \epsilon_t} E_t [Y_{t+s}] \\
	&= (1-\beta) \sum_{s=0}^\infty \beta^s \\
	&= (1-\beta) \frac{1}{1-\beta} \\
	&= 1
\end{align*}
Thus the change in savings is
\[
\frac{\partial}{\partial \epsilon_t} E_t [Y_{t+s} - C_{t+s}] = 1 - 1 = 0
\]
Since the initial shock to income increases income permanently, i.e., income in every period increases by $\epsilon_t$, there is no desire on part of the consumer to alter their savings behavior. Consumption simply increases 1-for-1 with the income shock.

\subsubsection*{c) $\Delta(Y_t) = \phi\Delta(Y_{t-1}) + \epsilon_t$}
Note that \[
\Delta(Y_{t+s}) = \phi^s(\Delta(Y_t) + \epsilon_t)
\]
Using the fact that by definition
\begin{align*}
	Y_{t+s} - Y_{t-1} &= \Delta(Y_{t+s}) + \Delta(Y_{t+s-1}) + \Delta(Y_{t+s-2}) + \ldots + \Delta(Y_{t+1}) + \Delta(Y_{t}) \\
	&= \phi^s(\Delta(Y_t) + \epsilon_t) + \phi^{s-1}(\Delta(Y_t) + \epsilon_t) + \ldots + \phi(\Delta(Y_t) + \epsilon_t) + (\Delta(Y_{t-1}) + \epsilon_t)
\end{align*}
we have
\begin{align*}
	Y_{t+s} &= \sum_{k=0}^s \left[ \phi^{s-k} (\Delta(Y_t) + \epsilon_t) \right] + Y_{t-1} \\
	&= Y_{t-1} + \sum_{k=0}^s \left[ \phi^{k} \Delta(Y_t) \right] + \sum_{k=0}^s \left[ \phi^{k} \epsilon_t \right] \\
	&= Y_{t-1} + \frac{1 - \phi^{s+1}}{1 - \phi} \Delta(Y_t) + \frac{1 - \phi^{s+1}}{1 - \phi} \epsilon_t
\end{align*}
Hence
\[
\frac{\partial}{\partial \epsilon_t} E_t [Y_{t+s}] = \frac{1 - \phi^{s+1}}{1 - \phi}
\]
We can then use this to obtain
\begin{align*}
	\frac{\partial}{\partial \epsilon_t} C_t &= (1-\beta) \sum_{s=0}^\infty \beta^s \frac{\partial}{\partial \epsilon_t} E_t [Y_{t+s}] \\
	&= (1-\beta) \sum_{s=0}^\infty \beta^s \frac{1 - \phi^{s+1}}{1 - \phi} \\
	&= (1-\beta) \sum_{s=0}^\infty \frac{\beta^s}{1 - \phi} - (1-\beta) \sum_{s=0}^\infty \frac{\beta^s\phi^{s+1}}{1 - \phi} \\
	&= \frac{1}{1-\phi} - \frac{(1-\beta) \phi}{(1-\phi)(1-\beta\phi)}
\end{align*}
Hence the change in savings is
\[
\frac{\partial}{\partial \epsilon_t} E_t [Y_{t+s} - C_{t+s}] = \frac{1 - \phi^{s+1}}{1 - \phi} - \frac{1}{1-\phi} + \frac{(1-\beta) \phi}{(1-\phi)(1-\beta\phi)}
\]
Initial change in savings is small or negative, while future change in savings in large. This is because the initial shock to income increases future income more than present income (the initial shock is amplified over time). Smooth consumption implies future savings should increase more than current savings.

\subsection{Discuss permanent income hypothesis.} In this model, consumers spread out any temporary gains in income over all time periods. In (a), the initial temporary gain in income leads to a large increase in initial savings that decreases over time, while in (c) we get the reverse: a large increase in future income causes savings to increase over time. However, in (b), a permanent income change that is uniform across all periods causes no change in savings. In all 3 cases, bumps in income translate into smooth (uniform across periods) changes in consumption.

\subsection{Smooth consumption puzzle.}
No, this is not a puzzle: If consumers want to smooth consumption (due to risk aversion for example), then transitory shocks in income should not lead to large changes in consumption.

\section{Estimation of adjustment costs}

\subsection{Optimality conditions}

First we write the Lagrangian
\[
\mathcal{L} = E \left\{ \sum_{t=0}^{\infty} R_t \left[(1 - \tau) K_t^{\alpha} L_t^{1-\alpha} - w_t L_t - I_t \left(1 + a(I_t / K_t - \delta) \right) \right] + q_t ((1-\delta) K_t + I_t - K_{t+1}) \right\}
\]
which gives first order conditions with respect to $L_t$, $I_t$, and $K_{t+1}$ (which are chosen each period) of
\[
(1-\alpha)(1 - \tau) K_t^{\alpha} L_t^{-\alpha} = w_t
\]
for $L_t$,
\[
q_t = 1 + a(I_t / K_t - \delta) + a \cdot \frac{I_t}{K_t}
\]
for $I_t$,
\[
R_t q_t = E \left[ R_{t+1} (\alpha (1-\tau) K_{t+1}^{\alpha-1} L_{t+1}^{1-\alpha} + a(I_{t+1} / K_{t+1})^2 + (1-\delta) q_{t+1} )\right]
\]
for $K_{t+1}$.

Optimal level of investment requires the capital stock and the value for $a$, as we can rewrite the investment optimality condition as
\[
q_t = 2a \frac{I_t}{K_t} + 1 - \delta a
\]


Log-linearizing the first order condition for investment, with $\delta = \frac{\bar{I}}{\bar{K}}$, we have
\[
\bar{q} = \delta a + 1 \implies \check{q} = \frac{2 \delta a}{1 + \delta a} \times (\check{I} - \check{K})
\]
Estimating $a$ using this equation would be simply a matter of regressing estimates of $q$ deviations on investment and capital deviations, and then solving for $a$ given that the coefficient would be $\frac{2 \delta a}{1 + \delta a}$. However, the error term would also be picking up transitions to a new steady state in addition to mere measurement error, and since one might expect investment and capital to rise over time, the error term would likely be correlated with growth and would be larger (and serially correlated) in times of more rapid growth.

\subsection{Euler equation}

Substituting the investment optimality condition and the definition of $R$, we have
\begin{dmath*}
	2a \frac{I_t}{K_t} + 1 - \delta a = E\left[ \frac{1}{1+r_{t+1}} \times (\alpha (1-\tau) K_{t+1}^{\alpha-1} L_{t+1}^{1-\alpha} + a(I_{t+1} / K_{t+1})^2 + (1-\delta) (2a \frac{I_{t+1}}{K_{t+1}} + 1 - \delta a) )\right]
\end{dmath*}
and we can proceed with log linearization.

The steady state value of $I/K$ is just $\delta$, so we have
\[
\frac{2 \delta a}{1 + \delta a} (\check{I}_t - \check{K}_t)
\]
for the left hand side.

For the right hand side, we get
\[
\frac{2 \delta a}{(1+\bar{r})(1+ \delta a)} E[ \check{I}_{t+1} - \check{K}_{t+1} ] + \frac{\alpha (\alpha - 1) (1-\tau)}{(1+\bar{r})(1+ \delta a)} \left( \frac{\bar{K}}{\bar{L}} \right)^{\alpha -1} E[ \check{K}_{t+1} - \check{L}_{t+1} ]
- \frac{\bar{r}}{1+\bar{r}} E[\check{r}_{t+1}]
\]


\subsection{Regression to estimate Euler equation.}
Denote the expectational error term as
\[
\epsilon_{\check{x}_{t+1}} = \check{x}_{t+1} - E_t[\check{x}_{t+1}]
\]
and define
\[
\bar{X} = (1-\tau)\alpha\left( \frac{\bar{L}}{\bar{K}} \right)^{1-\alpha} + \alpha \delta^2 + (1-\delta)(1+a\delta)
\]
and rearrange the log-linearized Euler equation with expectational error terms to be
\begin{dmath*}
	\check{r}_{t+1} = \frac{1+\bar{r}}{\bar{r}\bar{X}} (1-\tau) \alpha (1-\alpha) \left( \frac{\bar{L}}{\bar{K}} \right)^{1-\alpha} (\check{L}_{t+1} - \check{K}_{t+1}) + \frac{1+\bar{r}}{\bar{r}\bar{X}} 2 \alpha \delta (\check{I}_{t+1} - \check{K}_{t+1}) - \frac{1+\bar{r}}{\bar{r}} \frac{2a\delta}{1+a\delta} (\check{I}_{t} - \check{K}_{t}) - \frac{1+\bar{r}}{\bar{r}\bar{X}} (1-\tau) \alpha (1-\alpha) \left( \frac{\bar{L}}{\bar{K}} \right)^{1-\alpha} (\epsilon_{\check{L}_{t+1}} - \epsilon_{\check{K}_{t+1}}) - \frac{1+\bar{r}}{\bar{r}\bar{X}} 2 \alpha \delta (\epsilon_{\check{I}_{t+1}} - \epsilon_{\check{K}_{t+1}}) + \epsilon_{\check{r}_{t+1}} \\
	= \beta_0 (\check{L}_{t+1} - \check{K}_{t+1}) + \beta_1 (\check{I}_{t+1} - \check{K}_{t+1}) + \beta_2 (\check{I}_{t} - \check{K}_{t}) + u_{t+1}
\end{dmath*}
which is in the form of a linear model of observed variables, with
\begin{align*}
	\beta_0 &= \frac{1+\bar{r}}{\bar{r}\bar{X}} (1-\tau) \alpha (1-\alpha) \left( \frac{\bar{L}}{\bar{K}} \right)^{1-\alpha} \\
	\beta_1 &= \frac{1+\bar{r}}{\bar{r}\bar{X}} 2 \alpha \delta \\
	\beta_2 &= -\frac{1+\bar{r}}{\bar{r}} \frac{2a\delta}{1+a\delta} \\
	u_{t+1} &= \frac{1+\bar{r}}{\bar{r}\bar{X}} (1-\tau) \alpha (1-\alpha) \left( \frac{\bar{L}}{\bar{K}} \right)^{1-\alpha} (\epsilon_{\check{L}_{t+1}} - \epsilon_{\check{K}_{t+1}}) - \frac{1+\bar{r}}{\bar{r}\bar{X}} 2 \alpha \delta (\epsilon_{\check{I}_{t+1}} - \epsilon_{\check{K}_{t+1}}) + \epsilon_{\check{r}_{t+1}}
\end{align*}
Since $u_{t+1}$ consists of just constants multiplied with expectational error terms,
\[
E[u_{t+1} | \check{L}_{t+1}, \check{I}_{t+1}, \check{K}_{t+1}, \check{I}_{t}, \check{K}_{t}] = 0
\]
Hence by regressing $\check{r}_{t+1}$ on $(\check{L}_{t+1} - \check{K}_{t+1})$, $(\check{I}_{t+1} - \check{K}_{t+1})$, and $(\check{I}_{t} - \check{K}_{t})$, we can obtain estimates for $\beta_0$, $\beta_1$, and $\beta_2$. Given that the only unknown parameters are $\alpha$, $\delta$, and $a$, our model is just identified and we can obtain an estimate for $a$.

\subsection{Immediate capital stock adjustment.}
If capital adjusts instantly, the timing on the Euler equation changes, and the log-linearized Euler equation changes to become
\begin{dmath*}
	\frac{2a\delta}{\bar{X}}(\check{I}_t - \check{K}_t) - \frac{1}{\bar{X}}(1-\tau)\alpha \left( \frac{\bar{L}}{\bar{K}} \right)^{1-\alpha} (1-\alpha) (\check{L}_t - \check{K}_t) + \frac{2a\delta^2}{\bar{X}}(\check{I}_t - \check{K}_t) = E_t \left[\frac{2a\delta}{1+a\delta} (\check{I}_{t+1} - \check{K}_{t+1}) - \frac{\bar{r}}{1+\bar{r}} \check{r}_{t+1} \right]
\end{dmath*}
There should be less expectation error. We should still be able to estimate $a$ by rearranging this log-linearized Euler equation except with different timings on the variables.

\subsection{Estimating investment as a function of $q$ vs Euler equation.}
The former approach requires some assumptions about what $q$ is given that we cannot observe $q$. While $q$ should be $1$ when firms are not adjusting their capital stock, this is not always the case, e.g., during the period immediately after a productivity shock.

\section{Practice log-linearization}

\subsection{}
The original equation is
\[
Y_t = C_t + I_t + G_t + NX_t
\]
so we take logs to get
\[
\log(Y_t) = \log(C_t + I_t + G_t + NX_t)
\]
and do the first order Taylor series expansion at the steady state to get
\[
\cancel{\log \bar{Y}} + \frac{1}{\bar{Y}} (Y_t - \bar{Y}) = \cancel{\log(\overline{C_t + I_t + G_t + NX_t})} + \frac{1}{\bar{Y}} (C_t - \bar{C}) + \frac{1}{\bar{Y}} (I_t - \bar{I}) + \frac{1}{\bar{Y}} (G_t - \bar{G}) + \frac{1}{\bar{Y}} (NX_t - 0)
\]
and we rewrite as
\[
\check{y} = \frac{\bar{C}}{\bar{Y}} \frac{(C_t - \bar{C})}{\bar{C}} + \frac{\bar{I}}{\bar{Y}} \frac{(I_t - \bar{I})}{\bar{I}} + \frac{\bar{G}}{\bar{Y}} \frac{(G_t - \bar{G})}{\bar{G}} + \cancel{\frac{\overline{NX}}{\bar{Y}} \frac{(NX_t - \overline{NX})}{\overline{NX}}}
\]
which comes out to just
\[
\check{Y} = \frac{\bar{C}}{\bar{Y}} \check{C} + \frac{\bar{I}}{\bar{Y}} \check{Y} + \frac{\bar{G}}{\bar{Y}} \check{G}
\]

\subsection{}
The original equation is
\[
Y_t = (\alpha K_t^{\rho} + (1-\alpha)(A_t L_t)^{\rho})^{1 \over \rho}
\]
so we take logs to get
\[
\log(Y_t) = \log((\alpha K_t^{\rho} + (1-\alpha)(A_t L_t)^{\rho})^{1 \over \rho})
\]
and do the first order Taylor series expansion at the steady state to get
\begin{tiny}
\[
\check{Y}  = \left(\frac{\alpha  \bar{K}^{\rho -1}}{\alpha  \bar{K}^{\rho }-(\alpha -1) (\bar{A} \bar{L})^{\rho }}\right) (K_t - \bar{K}) + \left(\frac{(\alpha -1) (\bar{A} \bar{L})^{\rho }}{\bar{A} \left((\alpha -1) (\bar{A} \bar{L})^{\rho }-\alpha  \bar{K}^{\rho }\right)}\right) (A_t - \bar{A}) + \left(\frac{(\alpha -1) (\bar{A} \bar{L})^{\rho }}{\bar{L} \left((\alpha -1) (\bar{A} \bar{L})^{\rho }-\alpha  \bar{K}^{\rho }\right)}\right)(L_t - \bar{L})
\]
\end{tiny}
and rewrite as
\[
\check{Y}  = \left(\frac{\alpha  \bar{K}^{\rho}}{\alpha  \bar{K}^{\rho }-(\alpha -1) (\bar{A} \bar{L})^{\rho }}\right) \check{K} + \left(\frac{(\alpha -1) (\bar{A} \bar{L})^{\rho }}{\left((\alpha -1) (\bar{A} \bar{L})^{\rho }-\alpha  \bar{K}^{\rho }\right)}\right) \check{A} + \left(\frac{(\alpha -1) (\bar{A} \bar{L})^{\rho }}{\left((\alpha -1) (\bar{A} \bar{L})^{\rho }-\alpha  \bar{K}^{\rho }\right)}\right) \check{L}
\]


\subsection{}
The original equation is
\[
K_t = (1-\delta) K_{t-1} + I_t - \psi \left(\frac{I_t}{K_{t-1}} - \delta \right)^2 I_t
\]
so we take logs to get
\[
\log(K_t) = \log((1-\delta) K_{t-1} + I_t - \psi \left(\frac{I_t}{K_{t-1}} - \delta \right)^2 I_t)
\]
and do the first order Taylor series expansion at the steady state to get
\begin{dmath*}
\check{K} = \left( \frac{-\delta +\frac{2 \bar{I}^2 \psi  (\bar{I}-\delta  \bar{K})}{K^3}+1}{-\frac{\bar{I} \psi  (\bar{I}-\delta  \bar{K})^2}{\bar{K}^2}\bar{I}-\delta  \bar{K}+\bar{K}} \right) (K_{t-1} - \bar{K}) + \left( \frac{3 \bar{I}^2 \psi -4 \delta  \bar{I} \bar{K} \psi +\bar{K}^2 \left(\delta ^2 \psi -1\right)}{\bar{I}^3 \psi -2 \delta  \bar{I}^2 \bar{K} \psi +\bar{I} \bar{K}^2 \left(\delta ^2 \psi -1\right)+(\delta -1) \bar{K}^3} \right) (I_t - \bar{I})
\end{dmath*}
and rewrite as
\begin{dmath*}
\check{K} = \left( -\delta +\frac{2 \bar{I}^2 \psi  (\bar{I}-\delta  \bar{K})}{K^3}+1 \right) \check{K}_{t-1} + \left( \frac{3 \bar{I}^3 \psi -4 \delta  \bar{I}^2 \bar{K} \psi + \bar{I} \bar{K}^2 \left(\delta ^2 \psi -1\right)}{\bar{I}^3 \psi -2 \delta  \bar{I}^2 \bar{K} \psi +\bar{I} \bar{K}^2 \left(\delta ^2 \psi -1 \right)+(\delta -1) \bar{K}^3} \right) \check{I}_t
\end{dmath*}

\subsection{}

Since this is just a version of 4.3 with an additional term that is a constant in the steady state, the answer is the same.

\subsection{}
The original equation is
\[
\exp(i_t) = \left(\frac{P_t}{P_{t-1}} \right)^{\phi_{\pi}} \left(\frac{Y_t}{Y_{t-1}} \right)^{\phi_y} \exp(\rho i_{t-1})
\]
so we take logs to get
\[
i_t = \phi_{\pi}(\log(P_t) - \log(P_{t-1})) + \phi_y (\log(Y_t) - \log(Y_{t-1})) + \rho i_{t-1}
\]
and do the first order Taylor series expansion at the steady state to get
\[
i_t - \bar{i} = \frac{\phi_{\pi}}{\bar{P}} (P_t - \bar{P}) - \frac{\phi_{\pi}}{\bar{P}} (P_{t-1} - \bar{P}) +\frac{\phi_y}{\bar{Y}} (Y_t - \bar{Y}) - \frac{\phi_y}{\bar{Y}} (Y_{t-1} - \bar{Y}) + \rho(i_{t-1} - \bar{i})
\]
and we can rewrite as
\[
\check{i}_t = \frac{\phi_{\pi}}{\bar{i}} (\check{P}_t - \check{P}_{t-1})  +\frac{\phi_y}{\bar{i}} (\check{Y}_t - \check{Y}_{t-1}) + \rho \check{i}_{t-1}
\]

\subsection{}

The original equation is
\[
A \times F(L) = \frac{\frac{\partial U}{\partial L}}{\frac{\partial U}{\partial C}}
\]
so we take logs to get
\[
\log A + \log F(L) = \log \frac{\partial U}{\partial L} - \log \frac{\partial U}{\partial C}
\]
and do the first order Taylor series expansion at the steady state to get
\[
\frac{1}{\bar{A}} (A_t - \bar{A}) + \frac{\overline{F'(L})}{\overline{F(L)}} (F(L_t) - \overline{F(L)}) = \frac{\overline{\frac{\partial^2 U}{\partial L^2} }}{\overline{\frac{\partial U}{\partial L} }} (\frac{\partial U}{\partial L} - \overline{\frac{\partial U}{\partial L} }) + \frac{\overline{\frac{\partial^2 U}{\partial C^2} }}{\overline{\frac{\partial U}{\partial C} }} (\frac{\partial U}{\partial C} - \overline{\frac{\partial U}{\partial C} })
\]
and we can rewrite as
\[
\check{A} + F'(\bar{L})F(\check{L}) = \frac{\partial^2 U(\bar{L})}{\partial L^2} \times \check{\frac{\partial U}{\partial L}} +  \frac{\partial^2 U(\bar{L})}{\partial C^2} \times \check{\frac{\partial U}{\partial C}}
\]
which implies
\[
\check{A} + \frac{\bar{L} \cdot F'(\bar{L}}{F(\bar{L})} \check{L} = \bar{U}_{LC} \cdot \bar{C} \cdot \check{C} \left( \frac{1}{\bar{U}_L} - \frac{1}{\bar{U}_C} \right) - \bar{U}_{LC} \cdot \bar{L} \cdot \check{L} \left( \frac{1}{\bar{U}_L} - \frac{1}{\bar{U}_C} \right)
\]

\subsection{}
The original equation is
\[
Y_t = K_t^{\alpha_t} L_t^{1-\alpha_t}
\]
so we take logs to get
\[
\log(Y_t) = \alpha_t \log(K_t) + (1-\alpha_t) \log(L_t)
\]
and do the first order Taylor series expansion at the steady state to get
\[
\check{Y} =  \frac{\bar{\alpha} }{\bar{K}} (K_t - \bar{K}) + \frac{1-\bar{\alpha} }{\bar{L}} (L_t - \bar{L}) + (\log(\bar{K}) - \log(\bar{L})) (\alpha_t - \bar{\alpha})
\]
and we can rewrite as
\[
\check{Y} = \bar{\alpha} \check{K} + (1-\bar{\alpha}) \check{L} + (\log(\bar{K}) - \log(\bar{L})) \check{\alpha}
\]



\end{document}
