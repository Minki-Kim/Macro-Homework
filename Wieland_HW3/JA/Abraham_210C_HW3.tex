\documentclass[11pt]{article}

\usepackage[margin=1in]{geometry}
\usepackage[backend=bibtex, style=authortitle, citestyle=authoryear-icomp, url=false]{biblatex}
\usepackage{amsmath, amssymb, amsthm, mathrsfs}
\usepackage{caption, graphicx}
\usepackage{secdot, sectsty}
\usepackage{bbm}
\usepackage{fancyhdr}
\usepackage{listings}
\usepackage{color}
\usepackage{enumitem}
\usepackage{booktabs}
\usepackage{hyperref}
\usepackage{inconsolata}

\newcommand{\inv}[1]{#1^{-1}}
\newcommand{\iid}{\text{i.i.d.}}
\newcommand{\bmat}[1]{\begin{bmatrix} #1 \end{bmatrix}}
\newcommand{\asconv}{\xrightarrow{a.s.}}
\newcommand{\pconv}{\xrightarrow{p}}
\newcommand{\dconv}{\xrightarrow{d}}
\newcommand{\msconv}{\xrightarrow{m.s.}}
\newcommand{\liminfty}{\lim_{n \to \infty}}
\newcommand*\diff{\mathop{}\!\mathrm{d}}
\newcommand{\lhood}{\mathcal{L}}
\renewcommand{\vec}[1]{\mathbf{#1}}

\newtheorem*{proposition}{Proposition}
\newtheorem*{claim}{Claim}

\let\oldforall\forall
\let\forall\undefined
\DeclareMathOperator{\forall}{\oldforall}
\DeclareMathOperator{\ev}{E}
\DeclareMathOperator{\var}{Var}
\DeclareMathOperator*{\argmax}{arg\,max}
\DeclareMathOperator*{\argmin}{arg\,min}

\lstset{
  basicstyle=\footnotesize\ttfamily,
  columns=fixed,
  fontadjust=true,
  basewidth=0.5em
}

\allsectionsfont{\rmfamily}
\sectionfont{\normalsize}
\subsectionfont{\normalfont\normalsize\selectfont\itshape}
\subsubsectionfont{\normalfont\normalsize\selectfont\itshape}

\sectiondot{subsection}

\renewcommand\thesection{\Roman{section}}
\renewcommand\thesubsection{\thesection.\Alph{subsection}}
\renewcommand\thesubsubsection{\thesubsection.\arabic{subsubsection}}

\linespread{1}
\pagestyle{fancy}
\rhead{Econ 210C Homework 3}
\lhead{Justin Abraham}

\begin{document}

\section{RBC with Variable Labor Supply}

    Table \ref{tab:varlabor} compares volatilities across different parameterizations of $\eta$ with what is observed in the data. Larger values of $\eta$ improve the fit of the model. There is greater persistence since more elastic labor supply means shocks affect hours and wages to a greater extent. Consumption and labor supply, however, remain excessively smooth. The classical RBC model implies a greater Frisch elasticity based on the data.

    \begin{table}[h]
    \caption{Comparing second moments of output, consumption, and labor supply}
    \label{tab:varlabor}
    \centering
    \begin{tabular}{p{2cm}p{2cm}p{2cm}p{2cm}p{2cm}}
    \hline
     & Data & $\eta = 0.5$ & $\eta = 1$ & $\eta = 2$ \tabularnewline
    \hline
    Consumption & 1.27 & 0.97 & 1.03 & 1.09 \tabularnewline
    Output & 1.72 & 1.56 & 1.67 & 1.79 \tabularnewline
    Hours & 1.59 & 0.24 & 0.41 & 0.60 \tabularnewline
    \hline
    \end{tabular}
    \end{table}

\section{RBC with Variable Capital Utilization}

    The firm solves the profit maximization problem.

        $$ \max_{\{N_t, I_t, U_t\}} \ev \sum_{t=0}^\infty \prod_{s=0}^t (1+r_s)^{-1} (Y_t - N_t w_t - I_t) $$
        $$ Y_t = (U_t K_t)^\alpha (Z_t N_t)^{1-\alpha} $$
        $$ K_{t+1} = (1-\delta(U_t))K_t + I_t $$

    We equalize prices of capital and the consumption good as in equilibrium.

    \begin{enumerate}

        \item The first order conditions are given by

            \begin{align*}
                \text{Labor demand:~} & w_t = (1-\alpha) Z_t^{1-\alpha} \bigg ( \frac{U_t K_t}{N_t} \bigg )^\alpha \\
                \text{Shadow value of capital:~} & q_t = 1 \\
                \text{Euler equation:~} & q_t = \ev (1+r_{t+1})^{-1} \bigg ( \alpha U_{t+1}^\alpha \bigg ( \frac{K_{t+1}}{Z_{t+1} N_{t+1}} \bigg )^{\alpha-1} + q_{t+1} (1-\delta_{t+1}) \bigg ) \\
                \text{Utilization:~} & \alpha K_t^\alpha \bigg ( \frac{U_t}{Z_t N_t} \bigg )^{\alpha-1} = q_t \delta'(U_t) K_t
            \end{align*}

            $$ 1 + r_{t+1} = \alpha U_{t+1}^\alpha \bigg ( \frac{K_{t+1}}{Z_{t+1} N_{t+1}} \bigg )^{\alpha-1} + 1 - \delta(U_{t+1}) $$

        Rental rate depends on the marginal product of capital and depreciation which are themselves both dependent on $U_t$.

        \item Utilization satisfies $\alpha K_t^\alpha \big ( \frac{U_t}{Z_t N_t} \big )^{\alpha-1} = q_t \delta'(U_t) K_t$.

            $$ \ln \alpha + \alpha \ln K_t + (\alpha - 1) (\ln U_t - \ln Z_t - \ln N_t) = \ln q_t + \ln \delta'(U_t) + \ln K_t $$
            $$ \ln \alpha + (\alpha - 1) (\ln K_t + \ln U_t - \ln Z_t - \ln N_t) = \ln q_t + \ln \delta'(U_t) $$
            $$ (\alpha - 1) (\check K_t + \check U_t - \check Z_t - \check N_t) = \check q_t + \frac{\delta''(\bar U)}{\delta'(\bar U)} (U_t - \bar U) $$
            $$ (\alpha - 1) (\check K_t + \check U_t - \check Z_t - \check N_t) = \check q_t + \frac{\delta''(\bar U) \bar U}{\delta'(\bar U)} \check U_t = \check q_t + \Delta $$
            $$ \check U_t = \frac{\check q_t + \Delta}{\alpha - 1} (\check Z_t + \check N_t - \check K_t) = \frac{1}{1+\Delta} (\check Y_t - \check K_t) $$

        The final equality uses $\check q_t = 0$ from the first order condition and log linearized production.

        \item We can use the previous derivation to reduce $\check Y_t$ as a function of technology and inputs.

            $$ \check Y_t = \alpha (\check U_t + \check K_t) + (1-\alpha) (\check Z_t + \check N_t) $$
            $$ = \frac{\alpha}{1+\Delta} (\check Y_t - \check K_t) + \alpha \check K_t + (1-\alpha) (\check Z_t + \check N_t) $$
            $$ \frac{1-\alpha+\Delta}{1+\Delta} \check Y_t = \frac{\alpha \Delta}{1+\Delta} \check K_t + (1-\alpha) (\check Z_t + \check N_t) $$
            $$ \check Y_t = \frac{\alpha \Delta}{1-\alpha+\Delta} \check K_t + \frac{(1-\alpha) (1+\Delta)}{1-\alpha+\Delta} (\check Z_t + \check N_t) $$

        $\Delta$ governs the sensitivity of $\check U_t$ to the marginal rate of capital. The limiting case where $\Delta \to \infty$ is the standard (linearized) model and $\check U_t$ is fixed at full utilization. The limit $\delta \to 0$ represents the case of no utilization so that $\check Y_t$ depend solely on technology and labor.

        \item The linearized labor demand function is $\check w_t = \check Y_t - \check N_t$. Substituting the expression for $\check Y_t$,

            $$ \check w_t = \frac{\alpha \Delta}{1-\alpha+\Delta} \check K_t + \frac{(1-\alpha) (1+\Delta)}{1-\alpha+\Delta} (\check Z_t + \check N_t) - N_t $$

        We can obtain an upward sloping demand function if labor exhibits increasing returns to scale.

            $$ \frac{(1-\alpha) (1+\Delta)}{1-\alpha+\Delta} > 1 $$
            $$ (1-\alpha) (1+\Delta) > 1 - \alpha + \Delta $$
            $$ 1 - \alpha + \Delta - \Delta \alpha > 1 - \alpha + \Delta $$
            $$ - \Delta \alpha > 1 $$

        Since $\Delta, \alpha > 0$, indeterminacy is impossible in this model. One way to achieve indeterminacy is to incorporate positive production externalities so that aggregate labor has increasing returns to scale. A model with endogenous capital utilization is more likely to exhibit indeterminacy because it amplifies the importance of labor in production.

    \end{enumerate}

\section{Macroeconomics of Home Production}

    The household solves

        $$ \max_{c_m, c_h, L_h, L_m} (C_m^\rho + C_h^\rho)^\frac{1}{\rho} - \bigg ( \frac{1}{\eta}+1 \bigg )^{-1} (L_h + L_m)^{\frac{1}{\eta}+1} $$
        $$ \text{subject to~} C_m \leq w L_m, C_h \leq L_h $$

    The prices of consumption goods are equivalent.

    \begin{enumerate}

        \item The first order conditions are

            $$ \mathcal L = (C_m^\rho + C_h^\rho)^\frac{1}{\rho} - \bigg ( \frac{1}{\eta}+1 \bigg )^{-1} (L_h + L_m)^{\frac{1}{\eta}+1} + \lambda (w L_m - C_m) + \xi (L_h - C_h) $$
            $$ \lambda = (C_m^\rho + C_h^\rho)^\frac{1-\rho}{\rho} C_m^{\rho-1} $$
            $$ \xi = (C_m^\rho + C_h^\rho)^\frac{1-\rho}{\rho} C_h^{\rho-1} $$
            $$ (L_h + L_m)^\frac{1}{\eta} = \lambda w = \xi $$

        \item $w = \frac{\xi}{\lambda}$. Wage is the ratio of marginal utilities of market and home goods.

        \item $\lambda \frac{C_m}{C_h}^{1-\rho} = \xi$. The ratio of marginal utilities depends on the share of consumption.

        \item $C_h = L_h = C_m w^\frac{-1}{1-\rho}$. The share of consumption depends on the relative wage.

        \item Begin with $(L_h + L_m)^\frac{1}{\eta} = \lambda w$.

            $$ L_m = (\lambda w)^\eta - L_h $$
            $$ = (\lambda w)^\eta - C_m w^\frac{-1}{1-\rho} $$
            $$ = (\lambda w)^\eta - L_m w^\frac{-\rho}{1-\rho} $$
            $$ = (\lambda w)^\eta (1 + w^\frac{-\rho}{1-\rho})^{-1}  $$

        \item Differentiate the above expression.

            $$ \frac{\partial L_m}{\partial w} = \frac{(1 + W^{\frac{-\rho}{1-\rho}}) \lambda^{\eta} \eta W^{\eta - 1} - (\lambda W)^{\eta} (\frac{-\rho}{1-\rho}) W^{\frac{-\rho}{1-\rho} - 1}}{(1 + W^{\frac{-\rho}{1-\rho}})^2} $$

            $$ \varepsilon_m = \frac{\partial L_m}{\partial w} \frac{w}{L_m} = \frac{(1 + W^{\frac{-\rho}{1-\rho}}) \eta  -  (\frac{-\rho}{1-\rho}) W^{\frac{-\rho}{1-\rho}}}{(1 + W^{\frac{-\rho}{1-\rho}})} $$
        	$$ = \eta + \bigg ( \frac{\rho}{1-\rho} \bigg ) \bigg ( \frac{W^{\frac{-\rho}{1-\rho}}}{1 + W^{\frac{-\rho}{1-\rho}}} \bigg ) $$

        \item Observing labor supply volatilities, the model with home production implies $\eta$ weakly smaller than in the classical RBC because that model does not account for shifting ``employment'' in home production. With high elasticity of substitution, $\varepsilon_m \to \infty$ and households will more readily substitute market employment with home production. If goods are complements, $\varepsilon_m$ approaches the Frisch elasticity $\eta$.

        \item Using $\left( C_m^\rho + C_h^\rho \right)^{\frac{1}{\rho} -1}  C_m^{\rho-1} = \lambda$ we can substitute the budget constraints and $L_h = L_m W^{\frac{\rho}{\rho-1}}$ to get

            $$ ((W L_m)^{\rho} + L_h^{\rho})^{\frac{1}{\rho} -1} (W L_m)^{\rho-1} = \lambda $$

        	$$ ((W L_m)^{\rho} + ( W^{\frac{\rho^2}{\rho-1}}) L_m^{\rho})^{\frac{1}{\rho} -1} (W L_m)^{\rho-1} = \lambda $$

        Now the previous expression for $L_m$ becomes

        	$$ L_m = \frac{ \left[((W L_m)^{\rho} + ( W^{\frac{\rho^2}{\rho-1}}) L_m^{\rho})^{\frac{1}{\rho} -1} (W L_m)^{\rho-1}\right]^{\eta} W^{\eta}}{(1 + W^{\frac{\rho}{\rho-1}})} $$

        	$$ L_m = \frac{ \left[((W^{\rho} + W^{\frac{\rho^2}{\rho-1}}) L_m^{\rho})^{\frac{1}{\rho} -1} (W L_m)^{\rho-1}\right]^{\eta} W^{\eta}}{(1 + W^{\frac{\rho}{\rho-1}})} $$

        	$$ L_m = \frac{ \left[(W^{\rho} + W^{\frac{\rho^2}{\rho-1}})^{\frac{1}{\rho} -1} L_m^{1-\rho} (W L_m)^{\rho-1}\right]^{\eta} W^{\eta}}{(1 + W^{\frac{\rho}{\rho-1}})} $$

        	$$ L_m = \frac{ \left[(W^{\rho} + W^{\frac{\rho^2}{\rho-1}})^{\frac{1}{\rho} -1}  W ^{\rho-1}\right]^{\eta} W^{\eta}}{(1 + W^{\frac{\rho}{\rho-1}})} $$

        	$$ L_m = \left( 1 + W^{\frac{\rho}{\rho-1}} \right)^{\eta \left( \frac{1-\rho}{\rho} \right) -1} W^\eta $$

        \item Differentiate the expression.

        	\begin{equation*}
        	\frac{\partial L_m}{\partial W} = \eta W^{\eta -1 } \left( 1 + W^{\frac{\rho}{\rho-1}} \right)^{\eta \left( \frac{1-\rho}{\rho} \right) -1} + W^\eta \left( \eta \left( \frac{1-\rho}{\rho} \right) -1  \right)  \left( 1 + W^{\frac{\rho}{\rho-1}} \right)^{\eta \left( \frac{1-\rho}{\rho} \right) -2} \left( \frac{\rho}{\rho-1} \right) W^{\frac{1}{\rho-1}}
        	\end{equation*}

        	\begin{align*}
        	\frac{\partial L_m}{\partial W} \frac{W}{L_m} &= \eta + \left( \eta \left( \frac{1-\rho}{\rho} \right) -1  \right) \left( \frac{\rho}{\rho-1} \right) \frac{W^{\frac{\rho}{\rho-1}}}{1 + W^{\frac{\rho}{\rho-1}}} \\
        	& = \eta + \left(  \frac{\rho}{1-\rho} - \eta \right) \frac{W^{\frac{\rho}{\rho-1}}}{1 + W^{\frac{\rho}{\rho-1}}} \\
            & = \eta \left ( 1 - \frac{W^{\frac{\rho}{\rho-1}}}{1 + W^{\frac{\rho}{\rho-1}}} \right ) + \frac{\rho}{1-\rho} \frac{W^{\frac{\rho}{\rho-1}}}{1 + W^{\frac{\rho}{\rho-1}}}
        	\end{align*}

        This elasticity has a similar interpretation to the labor elasticity previously derived. The current elasticity is smaller because we control for wealth effects; it only reflects labor responses to relative marginal utilities. In other words, these are the elasticities for the uncompensated and compensated labor supply functions, respectively.

        \item Higher elasticity of substitution ($\rho \to 1$) results in higher ``total'' elasticity which can help explain large labor supply volatilities without an infeasibly high Frisch elasticity.

    \end{enumerate}

\section{$q$-Theory with Variable Capital Utilization}

    \begin{enumerate}

        \item The firm solves the profit maximization problem.

            $$ \max_{\{L_t, I_t, U_t\}} \ev \sum_{t=0}^\infty \prod_{s=0}^t (1+r_s)^{-1} \bigg [ Y_t - L_t w_t - I_t \bigg (1 + \phi \bigg ( \frac{I_t}{K_t} \bigg ) \bigg ) \bigg ] $$
            $$ Y_t = Z_t (U_t K_t)^\alpha L_t^{1-\alpha} $$
            $$ K_{t+1} = (1-\delta(U_t))K_t + I_t $$
            $$ Y_t = C_t + I_t $$

        We equalize prices of capital and the consumption good as in equilibrium. This is a dynamic problem because of the presence of adjustment costs. Firms face an intertemporal tradeoff when adjusting and utilizing capital stock due to this friction.

        \item The Lagrangian and first order conditions are as follows.

            $$ \mathcal L = \ev \sum_{t=0}^\infty \prod_{s=0}^t (1+r_s)^{-1} \bigg [ Z_t (U_t K_t)^\alpha L_t^{1-\alpha} - L_t w_t - I_t \bigg (1 + \phi \bigg ( \frac{I_t}{K_t} \bigg ) \bigg ) \bigg ] + q_t ((1-\delta(U_t))K_t + I_t - K_{t+1}) $$

            \begin{align*}
                \text{Labor demand:~} & w_t = (1-\alpha) \frac{Y_t}{L_t} \\
                \text{Shadow value of capital:~} & q_t = 1 + \phi \bigg ( \frac{I_t}{K_t} \bigg ) + \frac{I_t}{K_t} \phi' \bigg ( \frac{I_t}{K_t} \bigg ) \\
                \text{Euler equation:~} & q_t = \ev (1+r_{t+1})^{-1} \bigg ( \alpha \frac{Y_{t+1}}{K_{t+1}} + \frac{I_{t+1}^2}{K_{t+1}^2} \phi' \bigg ( \frac{I_t}{K_t} \bigg ) + q_{t+1} (1-\delta(U_{t+1})) \bigg ) \\
                \text{Utilization:~} & \alpha \frac{Y_t}{U_t} = q_t \delta'(U_t) K_t
            \end{align*}

        Taken together these define the user cost equivalence.

            $$ 1 + r_{t+1} = \alpha \frac{Y_{t+1}}{K_{t+1}} + \frac{I_{t+1}^2}{K_{t+1}^2} \phi' \bigg ( \frac{I_t}{K_t} \bigg ) + 1 + \delta(U_{t+1}) $$

        Rental rate depends on $U_t$ through depreciation and the marginal productivity of capital.

        \item Now log-linearize the FOC for utilization and solve.

            $$ \alpha \frac{Y_t}{U_t} = q_t \delta'(U_t) K_t $$
            $$ \ln \alpha + \ln Y_t - \ln U_t = \ln q_t + \ln \delta'(U_t) + \ln K_t $$
            $$ \check Y_t - \check U_t = \check q_t + \Delta \check U_t + \check K_t $$
            $$ \check U_t = \frac{\check Y_t - \check K_t - \check q_t }{1 + \Delta} $$

        $\check q_t$ affects the tradeoff the firm makes when employing capital. A higher value of capital $\check q_t$ makes the cost of utilization, through higher depreciation of capital, more severe.

        \item Firms will invest (divest) if $q$ is greater (less) than unity. Hence procyclicality of investment is tightly linked to procyclicality of $q$. This model predicts that utilization is \emph{countercyclical} since $U$ is inversely related to $q$. This reasoning is not unambiguously correct because it uses partial equilibrium logic. Cyclicality also depends on the marginal rate of capital $\check Y_t - \check K_t$, which can be procyclical due to labor supply shocks.

        \item Consider a positive productivity shock. Stylistically, $U$ will exhibit procyclicality if the shock is such that the new steady state capital stock is lower. This is because the marginal product of capital is enhanced by the productivity shock and a lower capital-labor ratio and because firms begin to divest. If the new steady state is higher, the higher rate of investment may offset the change in marginal product enough to make utilization countercyclical.

    \end{enumerate}

\section{Fiscal Multipliers in the RBC Model}

    The log-linearized system is

        \begin{align*}
            \text{Labor markets:~} & \check{C}_t + \frac{1}{\eta} \check{L}_t = \check{Y}_t - \check{L}_t \\
            \text{Asset markets:~} & \ev \check{C}_{t+1} - \check{C}_t = \frac{\alpha \frac{\bar{Y}}{\bar{K}}}{\alpha \frac{\bar{Y}}{\bar{K}} + (1-\delta)} (\ev \check{Y}_{t+1} - \check{K}_{t+1} ) \\
            \text{Capital law of motion:~} & \check{K}_{t+1} = (1-\delta) \check{K}_t + \delta \check{I}_t \\
            \text{Production:~} & \check{Y}_t = \alpha \check{K}_t + (1-\alpha) \check{L}_t \\
            \text{National accounting:~} & \check{Y}_t = \frac{\bar{C}}{\bar{Y}} \check{C}_t + \frac{\bar{I}}{\bar{Y}} \check{I}_t + \frac{\bar{G}}{\bar{Y}} \check{G}_t \\
            \text{Exogenous fiscal policy:~} & \check{G}_t = \rho_g \check{G}_{t-1} + \epsilon_t^g
        \end{align*}


    \begin{enumerate}

        \item We can reduce the system to two endogenous variables.

            $$ \check{K}_{t+1} = (1-\delta) \check{K}_t + \delta \check{I}_t $$
            $$ \check{K}_{t+1} = (1-\delta) \check{K}_t + \delta \check{I}_t $$

        Endogenous variables in reduced form are linear in $\check K_t$ and $\check G_t$.

            \begin{align*}
                \check C_t & = \nu_{CK} \check K_t + \nu_{CG} \check G_t \\
                \check K_{t+1} & = \nu_{KK} \check K_t + \nu_{KG} \check G_t
            \end{align*}

        \item Fix $Z=1$ and let $\rho_g = 0.9$.

    \end{enumerate}


\end{document}
