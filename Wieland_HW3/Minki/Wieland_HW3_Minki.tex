\documentclass[11pt]{amsart}
\usepackage{geometry}                % See geometry.pdf to learn the layout options. There are lots.
\geometry{a4paper}                   % ... or a4paper or a5paper or ...
%\geometry{landscape}                % Activate for for rotated page geometry
\usepackage[parfill]{parskip}    % Activate to begin paragraphs with an empty line rather than an indent
\usepackage{enumitem}
\usepackage{graphicx}
\usepackage{amssymb}
\usepackage{amsmath}
\usepackage{cancel}
\usepackage{epstopdf}
\DeclareGraphicsRule{.tif}{png}{.png}{`convert #1 `dirname #1`/`basename #1 .tif`.png}
\usepackage{breqn}
\usepackage{float}
\usepackage{breqn}

\title{Econ 210C Problem Set \# 3}
\author{Minki Kim}
%\date{}                                           % Activate to display a given date or no date

\begin{document}




\maketitle

\section{Variable labor supply in the RBC model}
\begin{figure}[H]
	\centering
	\includegraphics[width=\textwidth]{Q1}
	\caption{Impulse responses with varying $\eta$}
\end{figure}

\begin{table}[H]
	\centering
	\begin{tabular}{ccccc}
		\hline \hline 
		& $\eta=0.5$  & $\eta = 1$          & $\eta = 2$ & Data  \\
		\hline 
		$Stdev(Y)$ &  1.54    & 1.64    & 1.74    & 1.72     \\
		$Stdev(C)$ &  0.97    & 1.02   & 1.08       & 1.27 \\
		$Stdev(L)$ &   0.23   &  0.37   & 0.53     &  1.59 \\
		\hline
	\end{tabular}
	\caption{Response to a transitory discount factor shock}
\end{table}
As one would expect, the fits get better as we calibrate the Frisch elasticity to bigger values. A large Frisch elasticity generates stronger intertemporal substitution of labor suppply, and hence amplifies the effect of shocks. However, even with a large Frisch elasticity, consumption is too smooth, and the volatility of hours generated from the model falls short of the empirical counterpart. 

\section{Variable capital utilization in an RBC model}
\begin{enumerate}[label=(\alph*)]
	\item The Lagrangian of the firm's profit maximization problem is: 
	\begin{align*}
	\mathcal{L} =& \mathbb{E}_t \sum_{s} \left( \prod_{k=1}^{s} \left(1 + R_{t+k} \right) \right)^{-1} \\\
	 &\times \left[  \left( U_{t+s} K_{t+s-1}  \right)^{\alpha} \left(Z_{t+s} N_{t+s}  \right)^{1-\alpha}  - W_{t+s} N_{t+s} - I_{t+s} + q_{t+s} \left( -K_{t+s} + (1-\delta(U_t))K_{t+s-1} + I_{t+s} \right) \right] 
	\end{align*}
	The first order conditions are: 
	
	\begin{align*}
	\frac{\partial \mathcal{L}}{\partial N_{t}} : \quad & W_{t} = (1-\alpha) \frac{Y_t}{N_t} \\
	\frac{\partial \mathcal{L}}{\partial I_{t}} : \quad & q_t = 1 \\
	\frac{\partial \mathcal{L}}{\partial K_{t}} : \quad & q_t = \mathbb{E}_t \frac{1}{1 + R_{t+1}} \left[ \alpha \left( U_{t+1} K_{t}  \right)^{\alpha-1} U_{t+1} \left(Z_{t+1} N_{t+1}  \right)^{1-\alpha}  + q_{t+1} \left( 1-\delta(U_{t+1}) \right) \right] \\
		\frac{\partial \mathcal{L}}{\partial U_{t}} : \quad & q_t \delta^{'}(U_t) K_{t-1}  = \alpha \frac{Y_t}{U_t}
	\end{align*}
	
	Combining the second and the third equations, we get the expression for the rental rate of capital. 
	\begin{equation*}
	R_{t+1} = \alpha U_{t+1}^{\alpha} K_{t}^{\alpha-1}  \left(Z_{t+1} N_{t+1}  \right)^{1-\alpha} - \delta(U_{t+1})
	\end{equation*}
	The rental rate depends on $U_t$ because both $MPK$ and depreciation rates depend on $U_t$. 
	
	\item Log linearized version of $q_t \delta^{'}(U_t) K_{t-1}  = \alpha U_t^{\alpha -1} K_{t-1}^\alpha \left( Z_t N_t \right)^{1-\alpha}$ is
	\begin{align*}
	&\check{q_t}+ \frac{\delta^{''}(\bar{U}) \bar{U}}{\delta^{'}(\bar{U})} \check{U}_t + \check{K}_{t-1} = (\alpha -1) \check{U_t} + \alpha \check{K_{t-1}}  + (1-\alpha) \left(  \check{Z_t} + \check{N_t}\right) 
	\end{align*}
	Using $\check{q_t}=0$ and $\check{Y_t} = \alpha \left( \check{U_t} + \check{K_{t-1}} \right) + (1-\alpha) \left(  \check{Z_t} + \check{N_t}\right)$, we can express $\check{U_t}$ in terms of $\check{Y_t}, \check{K_t}$, and $\Delta$. 
	\begin{align*}
	\check{U_t} = \frac{1}{1 + \Delta} \left( \check{Y_t} - \check{K_{t-1}}\right)
	\end{align*}
	
	\item The production function in a log-linear form is: 
	\begin{align*}
	\check{Y_t} &= \alpha \left( \check{U_t} + \check{K_{t-1}} \right) + (1-\alpha) \left(  \check{Z_t} + \check{N_t}\right) \\
	& = \frac{\alpha}{1 + \Delta} \left( \check{Y_t} - \check{K_{t-1}} \right) + \alpha \check{K_{t-1}} + (1-\alpha ) \left(  \check{Z_t} + \check{N_t}\right)
	\end{align*} 
	Isolate $\check{Y_t}$:
	\begin{align*}
	\check{Y_t} &= \frac{\Delta \alpha }{1 + \Delta - \alpha} \check{K_{t-1}} + \frac{(1+\Delta)(1-\alpha)}{1+ \Delta - \alpha} \left(  \check{Z_t} + \check{N_t} \right)  \\
	& = \check{Z_t} + \check{N_t} \quad \left( \text{when } \Delta = 0 \right) \\
	& = \alpha \check{K_{t-1}} + (1-\alpha) \left( \check{Z_t } + \check{N_t} \right)   \quad \left( \text{when } \Delta = \infty \right)
	\end{align*}
	\begin{enumerate}[label = (\roman*)]
    \item $\Delta = 0 $ means that the steady state capital utilization rate is zero. Hence no matter how big the capital stock is, it does not contribute to the output. Therefore, deviations of output from its steady state solely depend on technology and labor. 
    
    \item $\Delta = \infty$ means that steady state capital utilization rate is one. In this case, this model boils down to a model without capital utilization, since 100\% of capital stock is always used in production. Therefore, deviations of output from its steady state depend on all three inputs of the production function, with weights corresponding to the inputs same as the Cobb-Douglas coefficients. 
    
    \item Consider the case when $0 < \Delta < \infty$. In this case, the log linearized production function is written as: 
    \begin{equation*}
    \check{Y_t} = \frac{\Delta \alpha }{1 + \Delta - \alpha} \check{K_{t-1}} + (1-\alpha) \left( \check{Z_t} + \check{N_t} \right)+ \frac{\alpha (1-\alpha)}{1+ \Delta - \alpha} \left(  \check{Z_t} + \check{N_t} \right)
    \end{equation*}
    Since capital stock is not fully used in production, the contributions of $Z_t$ and $N_t$ in $Y_t$ is higher than when $\Delta = \infty$. 
    \end{enumerate}
    \item 
\end{enumerate}

\section{Homework in macroeconomics}
\begin{enumerate}[label = (\alph*)]
	\item The Lagrangian for the household's maximization problem is:
	\begin{equation*}
	\mathcal{L} = \left( C_m^\rho + C_h^\rho \right)^{\frac{1}{\rho}} - \left( \frac{1}{\eta} + 1 \right)^{-1} \left(  L_h + L_m \right)^{\frac{1}{\eta} + 1} + \lambda \left( W L_m - C_m \right) + \xi \left(L_h - C_h \right)
	\end{equation*}
	The first order conditions for the interior solutions are:
	\begin{align*}
	\left( C_m^\rho + C_h^\rho \right)^{\frac{1}{\rho} -1} C_m^{\rho-1} &= \lambda \\
	\left( C_m^\rho + C_h^\rho \right)^{\frac{1}{\rho} -1} C_h^{\rho-1} & = \xi \\
	\left(  L_h + L_m \right)^{\frac{1}{\eta} }  &=\lambda W \\
	\left(  L_h + L_m \right)^{\frac{1}{\eta} }  &= \xi 	 
	\end{align*}
	\item $\xi = \lambda W $
	\item $\xi = \lambda \left( \frac{C_m}{C_h} \right)^{1-\rho}$ \\
	\item Assuming an interior solution, $C_h = L_h = C_m W^{\frac{1}{\rho-1}}$
	\item Combining $L_h = C_h = C_m W^{\frac{1}{\rho-1}}$ and $C_m = W L_m$, we get $L_h = L_m W^{\frac{\rho}{\rho-1}}$. Substituting this into $\left(  L_h + L_m \right)^{\frac{1}{\eta} }  =\lambda W$, we get:
	\begin{equation*}
	L_m \left( 1 + W^{\frac{\rho}{\rho-1}} \right) = \left( \lambda W \right)^\eta
	\end{equation*}
	Hence, 
	\begin{equation*}
	L_m = \frac{(\lambda W)^\eta}{1 + W^{\frac{\rho}{\rho-1}}}
	\end{equation*}
	
	\item Differentiate $L_m$ with respect to $W$:
	\begin{equation*}
	\frac{\partial L_h}{\partial W} = \frac{(1 + W^{\frac{\rho}{\rho-1}}) \lambda^{\eta} \eta W^{\eta - 1} - (\lambda W)^{\eta} (\frac{\rho}{\rho-1}) W^{\frac{\rho}{\rho-1} - 1}}{(1 + W^{\frac{\rho}{\rho-1}})^2}
	\end{equation*}
	Hence the elasticity of $L_m$ with respect to $W$ is: 
	\begin{align*}
	\varepsilon_{L_m,W} = \frac{\partial L_m}{\partial W} \cdot \frac{W}{L_m} &= \frac{(1 + W^{\frac{\rho}{\rho-1}}) \eta  -  (\frac{\rho}{\rho-1}) W^{\frac{\rho}{\rho-1}}}{(1 + W^{\frac{\rho}{\rho-1}})} \\
	&= \eta + \left( \frac{\rho}{1-\rho} \right) \left( \frac{W^{\frac{\rho}{\rho-1}}}{1 + W^{\frac{\rho}{\rho-1}}} \right) 
	\end{align*}
	
	\item Consider a case where $\rho \rightarrow 1$ ($C_m$ and $C_h$) being perfect substitutes. Then, $\frac{\rho}{1 - \rho} \rightarrow \infty$, pushing up the Frisch elasticity to infinity. Intuitively, if the household can home-produce everything on the market, there is no reason to supply labor to the market when the wage is lower than the value of the home produced goods. 
	
	As $\rho$ gets smaller, the Frisch elasticity also approaches to $\eta$. If home produced goods and goods on the market are not substitutable, then the Frisch elasticity is exactly equals to $\eta$. 
	\item
	We had
	\[
	\left( C_m^\rho + C_h^\rho \right)^{\frac{1}{\rho} -1}  C_m^{\rho-1} = \lambda
	\]
	so substitute the budget constraints
	\[
	((W L_m)^{\rho} + L_h^{\rho})^{\frac{1}{\rho} -1} (W L_m)^{\rho-1} = \lambda
	\]
	and use the substitution
	\[
	L_h = L_m W^{\frac{\rho}{\rho-1}}
	\]
	to get
	\[
	((W L_m)^{\rho} + ( W^{\frac{\rho^2}{\rho-1}}) L_m^{\rho})^{\frac{1}{\rho} -1} (W L_m)^{\rho-1} = \lambda
	\]
	so now substitute back in to
	\[
	L_m = \frac{(\lambda W)^{\eta}}{(1 + W^{\frac{\rho}{\rho-1}})}
	\]
	and we have
	\[
	L_m = \frac{ \left[((W L_m)^{\rho} + ( W^{\frac{\rho^2}{\rho-1}}) L_m^{\rho})^{\frac{1}{\rho} -1} (W L_m)^{\rho-1}\right]^{\eta} W^{\eta}}{(1 + W^{\frac{\rho}{\rho-1}})}
	\]
	and we can simplify to get
	\[
	L_m = \frac{ \left[((W^{\rho} + W^{\frac{\rho^2}{\rho-1}}) L_m^{\rho})^{\frac{1}{\rho} -1} (W L_m)^{\rho-1}\right]^{\eta} W^{\eta}}{(1 + W^{\frac{\rho}{\rho-1}})}
	\]
	and again to get
	\[
	L_m = \frac{ \left[(W^{\rho} + W^{\frac{\rho^2}{\rho-1}})^{\frac{1}{\rho} -1} L_m^{1-\rho} (W L_m)^{\rho-1}\right]^{\eta} W^{\eta}}{(1 + W^{\frac{\rho}{\rho-1}})}
	\]
	and the $L_m$ terms on the right side cancel so we have
	\[
	L_m = \frac{ \left[(W^{\rho} + W^{\frac{\rho^2}{\rho-1}})^{\frac{1}{\rho} -1}  W ^{\rho-1}\right]^{\eta} W^{\eta}}{(1 + W^{\frac{\rho}{\rho-1}})}
	\]
	Simplify this a little bit more:
	\begin{equation*}
	L_m = \left( 1 + W^{\frac{\rho}{\rho-1}} \right)^{\eta \left( \frac{1-\rho}{\rho} \right) -1} W^\eta
	\end{equation*}
	\item Differentiate $L_m$ w.r.t $W$
	\begin{equation*}
	\frac{\partial L_m}{\partial W} = \eta W^{\eta -1 } \left( 1 + W^{\frac{\rho}{\rho-1}} \right)^{\eta \left( \frac{1-\rho}{\rho} \right) -1} + W^\eta \left( \eta \left( \frac{1-\rho}{\rho} \right) -1  \right)  \left( 1 + W^{\frac{\rho}{\rho-1}} \right)^{\eta \left( \frac{1-\rho}{\rho} \right) -2} \left( \frac{\rho}{\rho-1} \right) W^{\frac{1}{\rho-1}}
	\end{equation*}
	Hence the elasticity of $L_m$ w.r.t $W$ is
	\begin{align*}
	\frac{\partial L_m}{\partial W} \frac{W}{L_m} &= \eta + \left( \eta \left( \frac{1-\rho}{\rho} \right) -1  \right) \left( \frac{\rho}{\rho-1} \right) \frac{W^{\frac{\rho}{\rho-1}}}{1 + W^{\frac{\rho}{\rho-1}}} \\
	& = \eta + \left(  \frac{\rho}{1-\rho} - \eta \right) \frac{W^{\frac{\rho}{\rho-1}}}{1 + W^{\frac{\rho}{\rho-1}}}
	\end{align*}
	Compare this result to the previous result in $(f)$:
	\begin{equation*}
	\frac{\partial L_m}{\partial W} \frac{W}{L_m} = \eta + \left( \frac{\rho}{1-\rho} \right) \left( \frac{W^{\frac{\rho}{\rho-1}}}{1 + W^{\frac{\rho}{\rho-1}}} \right)
	\end{equation*}
	\item

\end{enumerate}

\section{$q$-Theory with Variable Capital Utilization}
\begin{enumerate}[label = (\alph*)]
	\item The Lagrangian of the firm's profit maximization problem is: 
	\begin{align*}
	\mathcal{L} =& \mathbb{E}_t \sum_{s} \left( \prod_{k=1}^{s} \left(1 + r_{t+k} \right) \right)^{-1} \\
	\begin{split}
	&\times \left\lgroup  Z_{t+s} \left( U_{t+s} K_{t+s-1}  \right)^{\alpha} L_{t+s}^{1-\alpha}  - W_{t+s} L_{t+s} - I_{t+s} \left[ 1 + \phi \left( \frac{I_{t+s}}{K_{t+s-1}} \right) \right] \right. \\
	& \qquad + q_{t+s} \left( -K_{t+s} + (1-\delta(U_t))K_{t+s-1} + I_{t+s} \right) \left. \right\rgroup
	\end{split}
	\end{align*}
	This problem is truly dynamic because the presence of an adjustment cost links the present and future period investment decisions. 
	\item 
\end{enumerate}

\section{Fiscal multiplier in the RBC model}
\begin{enumerate}[label = (\alph*)]
	\item
	The log-linearized system of equations is
	\begin{align*}
		&\check{K}_t = (1-\delta) \check{K}_{t-1} + \delta \check{I}_t \\
		&\check{C}_t + \frac{1}{\eta} \check{L}_t = \check{Y}_t - \check{L}_t \\
		&E_t \check{C}_{t+1} - \check{C}_t = \frac{\alpha \frac{\bar{Y}}{\bar{K}}}{\alpha \frac{\bar{Y}}{\bar{K}} + (1-\delta)} (E_t \check{Y}_{t+1} - \check{K}_t ) \\
		&\check{Y}_t = \alpha \check{K}_{t-1} + (1-\alpha) \check{L}_t \\
		&\check{Y}_t = \frac{\bar{C}}{\bar{Y}} \check{C}_t + \frac{\bar{I}}{\bar{Y}} \check{I}_t + \frac{\bar{G}}{\bar{Y}} \check{G}_t \\
		&\check{G}_t = \rho_g \check{G}_{t-1} + \epsilon_t^g
	\end{align*}
	Guess that the policy functions take the form
	\begin{align*}
		&\check{C}_t = v_{CK} \check{K}_{t-1} + v_{CG} \check{G}_{t} \\
		&\check{K}_t = v_{KK} \check{K}_{t-1} + v_{KG} \check{G}_{t}
	\end{align*}
	Now plug the policy functions into the system of equations and we are left with the log-linearized consumption Euler equation and labor-leisure condition in terms of parameters, coefficients of the policy functions, and state variables:
	\begin{dmath*}
		(v_{CK} v_{KK} - v_{CK}) \check{K}_{t-1} + (v_{CK} v_{KG} + v_{CG}\rho_g - v_{CG}) \check{G}_{t} = \frac{\alpha \frac{\bar{Y}}{\bar{K}}}{\alpha \frac{\bar{Y}}{\bar{K}} + (1-\delta)} \left[ \frac{\bar{C}}{\bar{Y}} v_{CK} v_{KK} + \frac{\bar{I}}{\bar{Y}} (v_{KK} - 1 + \delta) v_{KK} \frac{1}{\delta} - v_{KK} \right] \check{K}_{t-1} + \frac{\alpha \frac{\bar{Y}}{\bar{K}}}{\alpha \frac{\bar{Y}}{\bar{K}} + (1-\delta)} \left[ \frac{\bar{C}}{\bar{Y}} v_{CK} v_{KG} + \frac{\bar{C}}{\bar{Y}} v_{CG} \rho_g + \frac{\bar{I}}{\bar{Y}} (v_{KK} - 1 + \delta) v_{KG} \frac{1}{\delta} + \frac{\bar{I}}{\bar{Y}} v_{KG} \rho_g \frac{1}{\delta} + \frac{\bar{G}}{\bar{Y}} \rho_g - v_{KG} \right] \check{G}_t
	\end{dmath*}
	\begin{dmath*}
		\left[ v_{CK} - \left( \frac{1}{\eta} + 1 \right) \frac{\alpha}{1-\alpha} \right] \check{K}_{t-1} + v_{CG} \check{G}_t = \frac{-\alpha - \frac{1}{\eta}}{1-\alpha} \left[ \frac{\bar{C}}{\bar{Y}} v_{CK} + \frac{\bar{I}}{\bar{Y}} (v_{KK} - 1 + \delta) \frac{1}{\delta} \right] \check{K}_{t-1} + \frac{-\alpha - \frac{1}{\eta}}{1-\alpha} \left[ \frac{\bar{C}}{\bar{Y}} v_{CG} + \frac{\bar{I}}{\bar{Y}} v_{KG} \frac{1}{\delta} + \frac{\bar{G}}{\bar{Y}} \right] \check{G}_{t}
	\end{dmath*}
	By comparing the coefficients on $\check{K}_{t-1}$ and $\check{G}_t$ in both equations, we obtain 4 equations
	\begin{dmath*}
		v_{CK} v_{KK} - v_{CK} = \frac{\alpha \frac{\bar{Y}}{\bar{K}}}{\alpha \frac{\bar{Y}}{\bar{K}} + (1-\delta)} \left[ \frac{\bar{C}}{\bar{Y}} v_{CK} v_{KK} + \frac{\bar{I}}{\bar{Y}} (v_{KK} - 1 + \delta) v_{KK} \frac{1}{\delta} - v_{KK} \right]
	\end{dmath*}
	\begin{dmath*}
		v_{CK} v_{KG} + v_{CG}\rho_g - v_{CG} = \frac{\alpha \frac{\bar{Y}}{\bar{K}}}{\alpha \frac{\bar{Y}}{\bar{K}} + (1-\delta)} \left[ \frac{\bar{C}}{\bar{Y}} v_{CK} v_{KG} + \frac{\bar{C}}{\bar{Y}} v_{CG} \rho_g + \frac{\bar{I}}{\bar{Y}} (v_{KK} - 1 + \delta) v_{KG} \frac{1}{\delta} + \frac{\bar{I}}{\bar{Y}} v_{KG} \rho_g \frac{1}{\delta} + \frac{\bar{G}}{\bar{Y}} \rho_g - v_{KG} \right]
	\end{dmath*}
	\begin{dmath*}
		v_{CK} - \left( \frac{1}{\eta} + 1 \right) \frac{\alpha}{1-\alpha} = \frac{-\alpha - \frac{1}{\eta}}{1-\alpha} \left[ \frac{\bar{C}}{\bar{Y}} v_{CK} + \frac{\bar{I}}{\bar{Y}} (v_{KK} - 1 + \delta) \frac{1}{\delta} \right]
	\end{dmath*}
	\begin{dmath*}
		v_{CG} \check{G}_t = \frac{-\alpha - \frac{1}{\eta}}{1-\alpha} \left[ \frac{\bar{C}}{\bar{Y}} v_{CG} + \frac{\bar{I}}{\bar{Y}} v_{KG} \frac{1}{\delta} + \frac{\bar{G}}{\bar{Y}} \right] \check{G}_{t}
	\end{dmath*}

	\item
	Increase in government expenditure is only temporary, and the boost in output is transitory. Households compensate for the increased government expenditure by decreasing consumption, decreasing savings, and increasing labor supply. The households desired level of steady-state capital remains unchanged. This causes MPL (and wages) to decline while MPK (and interest rate) to increase.

	\item
	The multiplier is about 1.13.

	\item
	Now the increase in government expenditure is permanent, and the boost in output persists forever. Households decrease consumption and increase labor supply by a lot more compared to before because they are increasing their savings in this situation. This is because the households want to attain a higher level of steady-state capital. This sharp increase in labor supply causes wages to decline by more and interest rate to increase more.
\end{enumerate}
\end{document}
