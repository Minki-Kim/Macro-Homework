\documentclass[11pt]{article}

\usepackage[margin=1in]{geometry}
\usepackage[backend=bibtex, style=authortitle, citestyle=authoryear-icomp, url=false]{biblatex}
\usepackage{amsmath, amssymb, amsthm, mathrsfs}
\usepackage{caption, graphicx}
\usepackage{secdot, sectsty}
\usepackage{bbm}
\usepackage{fancyhdr}
\usepackage{listings}
\usepackage{color}
\usepackage{enumitem}
\usepackage{booktabs}
\usepackage{hyperref}
\usepackage{inconsolata}
\usepackage{lmodern}

\newcommand{\inv}[1]{#1^{-1}}
\newcommand{\iid}{\text{i.i.d.}}
\newcommand{\bmat}[1]{\begin{bmatrix} #1 \end{bmatrix}}
\newcommand{\asconv}{\xrightarrow{a.s.}}
\newcommand{\pconv}{\xrightarrow{p}}
\newcommand{\dconv}{\xrightarrow{d}}
\newcommand{\msconv}{\xrightarrow{m.s.}}
\newcommand{\liminfty}{\lim_{n \to \infty}}
\newcommand*\diff{\mathop{}\!\mathrm{d}}
\newcommand{\lhood}{\mathcal{L}}
\renewcommand{\vec}[1]{\mathbf{#1}}

\newtheorem*{proposition}{Proposition}
\newtheorem*{claim}{Claim}

\let\oldforall\forall
\let\forall\undefined
\DeclareMathOperator{\forall}{\oldforall}
\DeclareMathOperator{\ev}{E}
\DeclareMathOperator{\var}{Var}
\DeclareMathOperator*{\argmax}{arg\,max}
\DeclareMathOperator*{\argmin}{arg\,min}

\lstset{
  basicstyle=\footnotesize\ttfamily,
  columns=fixed,
  fontadjust=true,
  basewidth=0.5em
}

\allsectionsfont{\rmfamily}
\sectionfont{\normalsize}
\subsectionfont{\normalfont\normalsize\selectfont\itshape}
\subsubsectionfont{\normalfont\normalsize\selectfont\itshape}

\sectiondot{subsection}

\renewcommand\thesection{\Roman{section}}
\renewcommand\thesubsection{\thesection.\Alph{subsection}}
\renewcommand\thesubsubsection{\thesubsection.\arabic{subsubsection}}

\linespread{1}
\pagestyle{fancy}
\rhead{Econ 210C Homework 4}
\lhead{Justin Abraham}

\begin{document}

\section{Labor Supply Problem}

    Households maximize lifetime utility.

        $$ \max_{\{C_t, L_t\}} = \sum_{t=0}^\infty \beta^t U(C_t, L_t) $$
        $$ \sum_{t=0}^\infty (1+r)^{-t} (C_t - w_t L_t) = 0 $$

    Assume $\beta = (1+r)^{-1}$ and wages follow a law of motion.

        $$ w_t = \begin{cases}
        w^H, & t = 1, 3, 5, \dots \\
        w^L, & t = 2, 4, 6, \dots
        \end{cases} $$

    We derive the labor supply functions for two variants of the utility function. First, let $U(C_t, L_t) = \log C_t + \log(1-L_t)$. The consumption-leisure condition is given by

        $$ \mathcal L = \sum_{t=0}^\infty (1+r)^{-t} [\log C_t + \log(1-L_t) - \lambda_t (C_t - w_t L_t)] $$
        $$ \frac{1}{C_t} w_t =  \frac{1}{1-L_t} $$
        $$ \hat L_t(w_t) = \frac{w_t - C_t}{w_t} $$

    Labor supply responds to shocks according to $\frac{\partial \hat L_t}{\partial w_t} = C_t w_t^{-2}$ and an elasticity of $\varepsilon_{\hat L} = C_t (L_t w_t)^{-1}$. Now let $U(C_t, L_t) = \log C_t + \log(1- 0.5 (L_t + L_{t-1}))$.

        $$ \mathcal L = \sum_{t=1}^\infty (1+r)^{-t} [\log C_t + \log(1- 0.5 (L_t + L_{t-1})) - \lambda_t (C_t - w_t L_t)] $$
        $$ \lambda_t = \frac{1}{C_t} $$
        $$ \lambda_t w_t = \frac{0.5}{1-L_t} + \frac{0.5}{1-L_t} $$

\section{Demand Shocks}

    An economy contains a continuum of identical consumers who solve

        $$ \max_{\{C_t, L_t\}} \ev \sum_{t=0}^\infty \beta^t \bigg ( \log C_t - v_t \frac{L_t^{1+\chi}}{1+\chi} \bigg ) $$
        $$ P_t K_{t+1} = W_t L_t + (P_t + d_t) K_t + \Pi_t - C_t $$
        $$ v_t \geq 0, \chi > 0 $$

    The capital stock is fixed at $\bar K$ and neither depreciates nor accumulates.

    \begin{enumerate}

        \item The consumption-leisure condition at time $t$ is $\frac{W_t}{C_t} = v_t L_t^\chi$.
        \item The consumer's intertemporal substitution satisfies $\frac{1}{C_t} = \frac{\beta}{C_{t+1} P_t} (P_{t+1} + d_{t+1})$.
        \item
        \item Labor supply is given by $L_t = \frac{W_t \lambda_t}{v_t}^\frac{1}{\chi}$ with elasticity $\frac{1}{\chi}$. A higher value of $v_t$ is associated with a higher marginal utility of leisure and implies a steeper labor supply curve.
        \item In this model, cyclicality is driven by demand shocks $v_t$.

    \end{enumerate}

\section{Business Cycle and External Returns to Scale}

    A continuum of competitive firms operate the production technology $Y_{it} = E_t K_{it}^\alpha (Z_t L_{it})^{1-\alpha}$. Firms take $E_t = Y_t^\frac{\gamma - 1}{\gamma}$ exogenously. The market clearing condition for consumption goods is $Y_t = C_t$.

    \begin{enumerate}

        \item The firm first order conditions yields the labor demand function.

            $$ \max_{K_{it}, L_{it}} E_t K_{it}^\alpha (Z_t L_{it})^{1-\alpha} - w_t L_t - d_t K_t $$

    \end{enumerate}

\section{Textbook Problems}

\end{document}
