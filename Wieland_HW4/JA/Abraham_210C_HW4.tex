\documentclass[11pt]{article}

\usepackage[margin=1in]{geometry}
\usepackage[backend=bibtex, style=authortitle, citestyle=authoryear-icomp, url=false]{biblatex}
\usepackage{amsmath, amssymb, amsthm, mathrsfs}
\usepackage{caption, graphicx}
\usepackage{secdot, sectsty}
\usepackage{bbm}
\usepackage{fancyhdr}
\usepackage{listings}
\usepackage{color}
\usepackage{enumitem}
\usepackage{booktabs}
\usepackage{hyperref}
\usepackage{inconsolata}
\usepackage{lmodern}

\newcommand{\inv}[1]{#1^{-1}}
\newcommand{\iid}{\text{i.i.d.}}
\newcommand{\bmat}[1]{\begin{bmatrix} #1 \end{bmatrix}}
\newcommand{\asconv}{\xrightarrow{a.s.}}
\newcommand{\pconv}{\xrightarrow{p}}
\newcommand{\dconv}{\xrightarrow{d}}
\newcommand{\msconv}{\xrightarrow{m.s.}}
\newcommand{\liminfty}{\lim_{n \to \infty}}
\newcommand*\diff{\mathop{}\!\mathrm{d}}
\newcommand{\lhood}{\mathcal{L}}
\renewcommand{\vec}[1]{\mathbf{#1}}

\newtheorem*{proposition}{Proposition}
\newtheorem*{claim}{Claim}

\let\oldforall\forall
\let\forall\undefined
\DeclareMathOperator{\forall}{\oldforall}
\DeclareMathOperator{\ev}{E}
\DeclareMathOperator{\var}{Var}
\DeclareMathOperator*{\argmax}{arg\,max}
\DeclareMathOperator*{\argmin}{arg\,min}

\lstset{
  basicstyle=\footnotesize\ttfamily,
  columns=fixed,
  fontadjust=true,
  basewidth=0.5em
}

\allsectionsfont{\rmfamily}
\sectionfont{\normalsize}
\subsectionfont{\normalfont\normalsize\selectfont\itshape}
\subsubsectionfont{\normalfont\normalsize\selectfont\itshape}

\sectiondot{subsection}

\renewcommand\thesection{\Roman{section}}
\renewcommand\thesubsection{\thesection.\Alph{subsection}}
\renewcommand\thesubsubsection{\thesubsection.\arabic{subsubsection}}

\linespread{1}
\pagestyle{fancy}
\rhead{Econ 210C Homework 4}
\lhead{Justin Abraham}

\begin{document}

\section{Labor Supply Problem}

    Households maximize lifetime utility.

        $$ \max_{\{C_t, L_t\}} = \sum_{t=0}^\infty \beta^t U(C_t, L_t) $$
        $$ \sum_{t=0}^\infty (1+r)^{-t} (C_t - w_t L_t) = 0 $$

    Assume $\beta = (1+r)^{-1}$ and wages follow a law of motion.

        $$ w_t = \begin{cases}
        w^H, & t = 1, 3, 5, \dots \\
        w^L, & t = 2, 4, 6, \dots
        \end{cases} $$

    We derive the labor supply functions for two variants of the utility function. First, let $U(C_t, L_t) = \log C_t + \log(1-L_t)$. The consumption-leisure condition is given by

        $$ \mathcal L = \sum_{t=0}^\infty (1+r)^{-t} [\log C_t + \log(1-L_t) - \lambda (C_t - w_t L_t)] $$
        $$ \frac{1}{C_t} w_t =  \frac{1}{1-L_t} $$
        $$ \hat L_t(w_t) = \frac{w_t - C_t}{w_t} $$

    Labor supply responds to shocks according to $\frac{\partial \hat L_t}{\partial w_t} = C_t w_t^{-2}$ in order to perfectly smooth consumption. Now let $U(C_t, L_t) = \log C_t + \log(1- 0.5 (L_t + L_{t-1}))$ and consider the nonnegativity constraint $L_t \geq 0$.

        $$ \mathcal L = \sum_{t=1}^\infty (1+r)^{-t} [\log C_t + \log(1- 0.5 (L_t + L_{t-1})) - \lambda (C_t - w_t L_t)] + \mu_t L_t $$
        $$ \lambda = \frac{1}{C_t} $$
        $$ \lambda_t w_t + \mu_t = \frac{0.5}{1-0.5(L_t + L_{t-1})} + \frac{0.5 (1+r)^{-t}}{1-0.5(L_{t+1} + L_t)} $$

    Since households anticipate the wage schedule, the first order condition for labor can be written as follows.

        $$ \lambda_t w_H + \mu_t = \frac{0.5}{1-0.5(L_H + L_L)} + \frac{0.5 (1+r)^{-t}}{1-0.5(L_L + L_H)} $$
        $$ \lambda_t w_L + \mu_t = \frac{0.5}{1-0.5(L_H + L_L)} + \frac{0.5 (1+r)^{-t}}{1-0.5(L_L + L_H)} $$

    These conditions are satisfied in the non-trivial case ($w_h \neq w_L$) if $\mu_t = 0$ when $t = 2, 4, 6, \dots$. This means that households will only work during high wage periods. Labor supply is much more elastic compared to the time-separable case since households only have preferences over the average labor supply over two periods. They experience no disutility from completely shifting labor to high return periods.

\section{Demand Shocks}

    An economy contains a continuum of identical consumers who solve

        $$ \max_{\{C_t, L_t\}} \ev \sum_{t=0}^\infty \beta^t \bigg ( \log C_t - v_t \frac{L_t^{1+\chi}}{1+\chi} \bigg ) $$
        $$ P_t K_{t+1} = W_t L_t + (P_t + d_t) K_t + \Pi_t - C_t $$
        $$ v_t \geq 0, \chi > 0 $$

    The capital stock is fixed at $\bar K$ and neither depreciates nor accumulates.

    \begin{enumerate}

        \item The consumption-leisure condition at time $t$ is $\frac{W_t}{C_t} = v_t L_t^\chi$.
        \item The consumer's intertemporal substitution satisfies $\frac{P_t}{C_t} = \ev \frac{\beta}{C_{t+1}} (P_{t+1} + d_{t+1})$.
        \item Since capital stock is fixed it must satisfy $\bar K = K_t = K_{t+1} = K^d \forall t$.
        \item Labor supply is given by $L_t = \frac{W_t \lambda_t}{v_t}^\frac{1}{\chi}$ with elasticity $\frac{1}{\chi}$. A higher value of $v_t$ is associated with a higher marginal utility of leisure and implies a steeper labor supply curve.
        \item In this model, it is not necessary for wages to be procycical in order for labor supply and consumption to be. Labor supply will contract in response to a positive demand shock $v_t$. If we embed this model of demand shocks into a RBC framework, we might imagine that dividends (through a lower marginal product of capital) and hence consumption will also exhibit cyclicality in response to a $v_t$ shock. One advantage of this modification is that consumption will be more procyclical since $v_t$ affects the household's labor-leisure tradeoff. A demand-driven business cycle has a disadvantage in forcing wages to be even more procyclical since there a demand shock is associated with a lower labor elasticity. We thus require a further modification to improve fit.

    \end{enumerate}

\section{Business Cycle and External Returns to Scale}

    A continuum of competitive firms indexed by $i$ operate the production technology $Y_{it} = E_t K_{it}^\alpha (Z_t L_{it})^{1-\alpha}$. Firms take $E_t = Y_t^\frac{\gamma - 1}{\gamma}$ exogenously. The market clearing condition for consumption goods is $Y_t = C_t$.

    \begin{enumerate}

        \item The firm first order conditions yields its factor demands.

            $$ \max_{K_{it}, L_{it}} E_t K_{it}^\alpha (Z_t L_{it})^{1-\alpha} - w_t L_{it} - d_t K_{it} $$
            $$ w_t = (1-\alpha) Y_{it} L_{it}^{-1} $$
            $$ d_t = \alpha Y_{it} K_{it}^{-1} $$

        \item With identical firms the aggregate production function is $Y_t = E_t K_t^\alpha (Z_t L_t)^{1-\alpha} = Y_t^\frac{\gamma - 1}{\gamma} K_t^\alpha (Z_t L_t)^{1-\alpha}$. Aggregate production exhibits increasing returns to scale $Y_t = [K_t^\alpha (Z_t L_t)^{1-\alpha}]^\gamma$ with $\gamma \geq 1$. Firms with increasing returns to scale in production make positive profit only if there is no markup? The social labor demand curve is $w_t = \gamma (1-\alpha) Y_t L_t^{-1}$.

        \item Consider a positive demand shock in $v_t$. The stylized facts are largely consistent with the model predictions. Consumption and labor supply are procyclical because the demand shock motivates a substitution towards leisure. Labor productivity will be slightly less procyclical due to increasing returns to scale in aggregate. Procyclicality of wages is less clear, as production externalities imply a flatter social labor demand but a demand shock causes the labor supply to become steeper.

        \item In the case where $\gamma = 1$ and $v_t$ is constant, we return to the standard RBC model with technology shocks. Co-movement of aggregate variables is consistent but consumption is excessively smooth and wages are too procyclical.

        \item We have argued that introducing production externalities allows wages to be less procyclical but that $v_t$ shocks achieve the opposite effect. It is conceivable to examine a business cycle driven by both $v_t$ and $Z_t$ such that demand shocks are not so prominent as to induce greater wage procyclicality. A higher value of $\gamma$ flattens the social labor demand function, reducing wage fluctuations so that we might not have to rely on technology shocks as much to control wage procyclicality.

    \end{enumerate}

\section{Textbook Problems}

    \begin{enumerate}

        \item Imperfectly competitive firms experience a loss $K(p_i - p^*)^2, K > 0$ from not adjusting prices. Let $p^* = p + \phi y$ and $y = m - p$. The fixed adjustment cost is $Z$. Suppose the initial state of the economy is $y = m = p = 0$ and $m$ changes to $m'$.

        \begin{enumerate}

            \item A fraction $f$ firms adjust prices so that the general price level is $p = f p^*$.

                \begin{align*}
                    p^* = & \frac{\phi m'}{1 + \phi f - f} \\
                    p = & \frac{f \phi m'}{1 + \phi f - f} \\
                    y = & m' - \frac{f \phi m'}{1 + \phi f - f}
                \end{align*}

            \item The loss from not adjusting is plotted as a function of $f$ in Figure \ref{fig:loss}.

            \begin{figure}[h]
                \centering
                \caption{Firm loss as a function of fraction adjusting}
                \label{fig:loss}
                \includegraphics[width=\textwidth]{loss.png}
            \end{figure}

            \item Suppose $\phi < 1$ so that the incentive to adjust is increasing in the fraction of firms adjusting. Fix $Z > K(\phi m')^2$ when $f = 0$ so that no adjustment is the Nash equilibrium. In the case when $Z < KK(\phi m')^2$, firms will profitably adjust even when no other firm adjusts. Full price adjustment is the Nash equilibrium here. Now let $\phi > 1$ and $Z = K \frac{\phi m'}{1 + \phi f - f}^2$ for some $f \in (0,1)$. Since the incentive is now decreasing as a function of $f$, only up to $f$ fraction of firms will adjust until the incentive is less than the menu cost. This will be a Nash equilibrium.

        \end{enumerate}

        \item The profit of an imperfectly competitive, representative firm is given by a concave function $\pi(y, r_i)$. Let $r^*(y)$ denote the profit-maximizing price set by the firm.

        \begin{enumerate}

            \item The firm's incentive to adjust its own price is $G = \pi(y_1, r^*(y_1)) - \pi(y_1, r^*(y_0))$ or the difference between firm profits under the new and old prices, given the new money supply and output.

            \item Consider the second-order Taylor approximation around $y_1$.

                $$ \pi(y_1, r^*(y_0)) = \pi(y_1, r^*(y_1)) + \pi'_r(y_1, r^*(y_1)) r^*(y_0) (y_1 - y_0) + \frac{1}{2} \pi''_r(y_1, r^*(y_1)) r^*(y_0)^2 (y_1 - y_0)^2 $$

            Applying the first order condition for real price and rearranging yields

                $$ G \approx -\frac{1}{2} \pi''_r(y_1, r^*(y_1)) r^*(y_0)^2 (y_1 - y_0)^2 $$

            \item $\pi''_r(y_1, r^*(y_1))$ is the sensitivity of the profit function to price changes while $r^*(y_0)^2$ is a measure of real rigidity, the responsiveness of real prices to output.

        \end{enumerate}

        \item Firm $i$ operates a linearized technology $y_i = s + \alpha \ell_i$. Prices are given by $p_i = w_i + (1-\alpha) \ell_i - s$. The aggregate output and price are $y = s + \alpha \ell$ and $p = w + (1-\alpha) \ell - s$. Wages are indexed to prices through $w = \theta p$. Aggregate demand is $y = m - p$. Supply shock $s$ and money supply $m$ are independent, mean-zero random variables.

        \begin{enumerate}

            \item

                $$ p = \frac{(1-\theta) m - s}{1-\theta \alpha} $$
                $$ y = \frac{(1-\theta) \alpha m + s}{1-\theta \alpha} $$
                $$ w = \theta \frac{(1-\theta) m - s}{1-\theta \alpha} $$

            The following second derivatives show that indexation reduces the impact of a monetary shock but aimplifies the impact of a supply shock.

                $$ \frac{\partial^2 \ell}{\partial m \partial \theta} = \frac{\alpha - 1}{(1-\theta \alpha)^2} $$
                $$ \frac{\partial^2 \ell}{\partial s \partial \theta} = \frac{1}{(1-\theta \alpha)^2} $$

            \item The variance of employment is minimized by $\theta^*$.

                $$ \min_\theta \var(\ell) = \left(\frac{1-\theta}{1-\theta \alpha}\right)^2 \sigma^2_m + \left(\frac{\theta}{1-\theta \alpha}\right)^2 \sigma^2_s $$
                $$ (1-\theta)(\alpha-1) \sigma^2_m + \theta \sigma^2_s $$
                $$ \theta^* = \frac{(1-\alpha) \sigma^2_m}{(1-\alpha) \sigma^2_m + \sigma^2_s} $$

            \item Suppose firm $i$ faces demand $y_i = y - \eta(p_i - p)$ and indexes wages to its own price by $\theta_i$. Substitute the reduced form expressions to obtain an expression for $\ell_i$.

                $$ \ell_i = \frac{(1-\theta) \alpha - \phi (1-\alpha) (\theta_i - \theta)) m + (\theta \alpha + \phi (\theta_i - \theta)) s}{(1-\theta \alpha) \alpha} $$

            The variance of firm $i$'s labor demand is minimized by $\theta_i^*$.

                $$ \var(\ell_i) = \left(\frac{(1-\theta) \alpha - \phi (1-\alpha) (\theta_i - \theta)}{(1-\theta \alpha) \alpha}\right)^2 \sigma^2_m + \left(\frac{\theta \alpha + \phi (\theta_i - \theta)}{(1-\theta \alpha) \alpha}\right)^2 \sigma^2_s $$

                $$ (\alpha - 1) \phi [(1-\theta) \alpha - \theta_i (1-\alpha) \phi + \theta (1-\alpha)] \sigma^2_m + \phi[\theta \alpha + \phi \theta_i - \phi \theta] \sigma^2_s = 0 $$

                $$ \theta_i^* = \frac{(1-\alpha) \phi ((1-\theta)\alpha + \theta \phi (1-\alpha)) \sigma^2_m + \phi \theta(\alpha - \phi) \sigma^2_s}{(\phi(1-\alpha))^2 \sigma^2_m + \phi^2 \sigma^2_s} $$

            The value of the index that satistifies $\theta_i = \theta$ is the Nash equilibrium value.

            $$ (\alpha - 1) \phi [(1-\theta) \alpha - \theta (1-\alpha) \phi + \theta (1-\alpha)] \sigma^2_m + \phi[\theta \alpha + \phi \theta - \phi \theta] \sigma^2_s = 0 $$

            $$ \theta_{\text{Nash}} = \frac{(1-\alpha) \sigma^2_m}{(1-\alpha) \sigma^2_m + \sigma^2_s} $$

        \end{enumerate}

    \end{enumerate}

\end{document}
