\documentclass[11pt]{amsart}
\usepackage{geometry}                % See geometry.pdf to learn the layout options. There are lots.
\geometry{a4paper}                   % ... or a4paper or a5paper or ...
%\geometry{landscape}                % Activate for for rotated page geometry
\usepackage[parfill]{parskip}    % Activate to begin paragraphs with an empty line rather than an indent
\usepackage{enumitem}
\usepackage{graphicx}
\usepackage{amssymb}
\usepackage{amsmath}
\usepackage{cancel}
\usepackage{epstopdf}
\DeclareGraphicsRule{.tif}{png}{.png}{`convert #1 `dirname #1`/`basename #1 .tif`.png}
\usepackage{breqn}
\usepackage{float}
\usepackage{breqn}

\title{Econ 210C Problem Set \# 4}
\author{Nathaniel Bechhofer}
%\date{}                                           % Activate to display a given date or no date

\begin{document}




\maketitle

\section{Labor supply problem}
\section{Demand shock}

\subsection*{(a)}

The consumption-leisure condition at time $t$ is just equating marginal benefits of labor and leisure $\frac{W_t}{C_t} = v_t L_t^\chi$.

\subsection*{(b)}

The consumer consuming one less unit today means they are losing out on $\frac{1}{C_t}$ today. 
With that, they can buy $\frac{1}{P_t}$ units of capital today, and tomorrow they will get $P_{t+1} + d_{t+1}$ from that unit of capital. 
So their payoff tomorrow is 
\[
\frac{1}{P_t} \times (P_{t+1} + d_{t+1}) \times \frac{\beta}{C_{t+1}}
\]
and optimality therefore implies we have
\[
\frac{1}{C_t} = \frac{1}{P_t} \times (P_{t+1} + d_{t+1}) \times \frac{\beta}{C_{t+1}}
\]
as the inter-temporal optimality condition.

\subsection*{(c)}



\section{Business cycle and external returns to scale}

\subsection*{(a)}

Each firm sets wage equal to marginal product of labor, so we have
\[
W_t = Y_t^{1-1/\gamma} \left(\frac{K_{it}}{L_{it}}\right)^{\alpha} Z_t^{1-\alpha}
\]
and so we can find labor demand as a function of wages
\[
L_{it} = (W_t Z_t^{\alpha-1} Y_t^{1/\gamma -1} K_{it}^{-\alpha})^{-\frac{1}{\alpha}}
\]
which simplifies to
\[
L_{it} = W_t^{-\frac{1}{\alpha}} Z_t^{\frac{1-\alpha}{\alpha}} Y_t^{\frac{1-1/\gamma}{\alpha}} K_{it}
\]

\subsection*{(b)}

Integrating both sides over all firms, we have
\[
L_t = W_t^{-\frac{1}{\alpha}} Z_t^{\frac{1-\alpha}{\alpha}} Y_t^{\frac{1-1/\gamma}{\alpha}} K_t
\]
so we can start to solve for aggregate production, so we get
\[
Y_t^{\frac{1-1/\gamma}{\alpha}} = \frac{L_t}{K_t} \times W_t^{\frac{1}{\alpha}} Z_t^{\frac{\alpha - 1}{\alpha}}
\]
and solving for $Y$ gives
\[
Y_t = \left(\frac{L_t}{K_t} \right)^{\frac{\alpha}{1-1/\gamma}} W_t^{\frac{1}{1-1/\gamma}} Z_t^{\frac{\alpha - 1}{1-1/\gamma}}
\]



\section{Problems from Romer}
\subsection{Problem 6.10}

\subsection{Problem 6.11}

\subsection{Problem 6.12}

\subsubsection*{(a)}
First, we can substitute our wage expression $w = \theta p$ to get
\[
p = \theta p + (1-\alpha) \ell - s \implies p = \frac{(1-\alpha) \ell - s}{1-\theta}
\]
and we are given aggregate demand, so we have
\[
y = m - p = m - \frac{(1-\alpha) \ell - s}{1-\theta}
\]
and from our output equation we have
\[
s + \alpha \ell = m - \frac{(1-\alpha) \ell - s}{1-\theta} \implies \ell = \frac{(1-\theta) m + \theta s}{1-\theta \alpha}
\]
which we can substitute into our price equation to get
\[
p = \frac{(1-\theta) m - s}{1-\theta \alpha}
\]
and now we have our output
\[
y = \frac{(1-\theta) \alpha m + s}{1-\theta \alpha}
\]
and we can find wage using the original wage expression to get
\[
w = \theta \times \frac{(1-\theta) m - s}{1-\theta \alpha}
\]
and we can now find how employment responds to shocks.

We can take mixed second derivatives to obtain
\[
\frac{\partial^2 \ell}{\partial m \partial \theta} = \frac{\alpha - 1}{(1-\theta \alpha)^2}
\]
for how indexation moderates the effect of a monetary shock and
\[
\frac{\partial^2 \ell}{\partial s \partial \theta} = \frac{1}{(1-\theta \alpha)^2}
\]
for how indexation moderates the effect of a supply shock.

Since $\alpha - 1 < 0$, we have that greater indexation reduces the effect of a monetary shock, while $1 > 0$ tell us that greater indexation scales the effects of supply shocks up.

\subsubsection*{(b)}

With independence we can just use the formula for the variance of a linear combination of two random variables. So we have
\[
\operatorname{Var}(\ell) = \left(\frac{1-\theta}{1-\theta \alpha}\right)^2 \operatorname{Var}(m) + \left(\frac{\theta}{1-\theta \alpha}\right)^2 \operatorname{Var}(s)
\]
so minimizing this requires the first order condition
\[
(1-\theta)(\alpha-1) \operatorname{Var}(m) + \theta \operatorname{Var}(s)
\]
so solving for $\theta$ we get
\[
\theta^* = \frac{(1-\alpha) \operatorname{Var}(m)}{(1-\alpha) \operatorname{Var}(m) + \operatorname{Var}(s)}
\]
as the wage indexation that minimizes employment variance.

\subsubsection*{(c.i)}

We can easily see that we have
\[
y_i - y = \alpha (\ell_i - \ell) \implies \ell_i = \ell + \frac{y_i - y}{\alpha} = \ell - \frac{\theta_i - \theta}{\alpha} \times \phi p
\]
and we already have expressions for employment and the price levels that we can substitute to get
\[
\ell_i = \frac{(1-\theta) \alpha - \phi (1-\alpha) (\theta_i - \theta)) m + (\theta \alpha + \phi (\theta_i - \theta)) s}{(1-\theta \alpha) \alpha}
\]
for employment at firm $i$.

\subsubsection*{(c.ii)}

We now have
\[
\operatorname{Var}(\ell_i) = \left(\frac{(1-\theta) \alpha - \phi (1-\alpha) (\theta_i - \theta)}{(1-\theta \alpha) \alpha}\right)^2 \operatorname{Var}(m) + \left(\frac{\theta \alpha + \phi (\theta_i - \theta)}{(1-\theta \alpha) \alpha}\right)^2 \operatorname{Var}(s)
\]
so we must satisfy the first order condition
\[
(\alpha - 1) \phi [(1-\theta) \alpha - \theta_i (1-\alpha) \phi + \theta (1-\alpha)] \operatorname{Var}(m) + \phi[\theta \alpha + \phi \theta_i - \phi \theta] \operatorname{Var}(s) = 0
\]
which allows us to solve for $\theta_i$ to get
\[
\theta_i^* = \frac{(1-\alpha) \phi ((1-\theta)\alpha + \theta \phi (1-\alpha)) \operatorname{Var}(m) + \phi \theta(\alpha - \phi) \operatorname{Var}(s)}{(\phi(1-\alpha))^2 \operatorname{Var}(m) + \phi^2 \operatorname{Var}(s)}
\]

\subsubsection*{(c.iii)}

The Nash equilibrium value implies that each firm's first order condition can have $\theta_i$ and $\theta$ identical. So we need 
\[
(\alpha - 1) \phi [(1-\theta) \alpha - \theta (1-\alpha) \phi + \theta (1-\alpha)] \operatorname{Var}(m) + \phi[\theta \alpha + \phi \theta - \phi \theta] \operatorname{Var}(s) = 0
\]
which simplifies to 
\[
(1-\theta) \alpha (1 - \alpha) \phi \operatorname{Var}(m) - \theta \alpha \phi \operatorname{Var}(s) = 0
\]
allowing us to solve for the Nash value of $\theta$ as
\[
\theta_{\text{Nash}} = \frac{(1-\alpha) \operatorname{Var}(m)}{(1-\alpha) \operatorname{Var}(m) + \operatorname{Var}(s)}
\]
which is the same value as in part b.

\end{document}
