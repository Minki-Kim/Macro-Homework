\documentclass[11pt]{amsart}


\usepackage{geometry}                % See geometry.pdf to learn the layout options. There are lots.
\geometry{a4paper}                   % ... or a4paper or a5paper or ...
%\geometry{landscape}                % Activate for for rotated page geometry
\usepackage[parfill]{parskip}    % Activate to begin paragraphs with an empty line rather than an indent
\usepackage{enumitem}
\usepackage{graphicx}
\usepackage{amssymb}
\usepackage{amsmath}
\usepackage{cancel}
\usepackage{epstopdf}
\DeclareGraphicsRule{.tif}{png}{.png}{`convert #1 `dirname #1`/`basename #1 .tif`.png}
\usepackage{breqn}

\title{Econ 210C Problem Set \# 2}
\author{Nathaniel Bechhofer}
%\date{}                                           % Activate to display a given date or no date

\begin{document}




\maketitle

\section{Investment and the Housing Market}

\subsection*{(a)}
\begin{enumerate}
	\item $I = \psi (P)$: Gross investment in housing is an increasing function of the price of houses. This specification implies that housing investment can be interpreted as the supply of new housing. 
	\item $r + \delta = (R + \dot{P})/P$: This implies that the costs of investing into a house, namely forgone investment income and depreciation are equal to the benefits, namely rental payments and capital gains. 
	\item $R = R(H)$: Rental cost is a decreasing function of the size of the housing stock. 
	\item $\dot{H} = I - \delta H$: The housing stock can change in two ways, housing investment and depreciation.
\end{enumerate}

\subsection*{(b)}

We merely substitute to obtain

\begin{align*}
\dot{H} &= \psi(P) - \delta H \\
r + \delta &= (R(H) + \dot{P}) / P
\end{align*}

\end{document}