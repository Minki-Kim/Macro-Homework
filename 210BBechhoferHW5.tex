\documentclass[11pt]{amsart}
\usepackage{geometry}                % See geometry.pdf to learn the layout options. There are lots.
\geometry{letterpaper}                   % ... or a4paper or a5paper or ... 
%\geometry{landscape}                % Activate for for rotated page geometry
\usepackage[parfill]{parskip}    % Activate to begin paragraphs with an empty line rather than an indent
\usepackage{graphicx}
\usepackage{amssymb}
\usepackage{epstopdf}
\usepackage{upgreek}
\DeclareGraphicsRule{.tif}{png}{.png}{`convert #1 `dirname #1`/`basename #1 .tif`.png}

\title{ECON 210B Homework \#5}
\author{Nathaniel Bechhofer}


\begin{document}
\maketitle

\section{Unemployment Insurance}

\subsection{Bellman equations}

From the notes we have
\begin{align*}
rU_t &= z + \dot{U}_t + \theta_t q(\theta_t) (W_t - U_t) \\
rW_t &= w_t + \dot{W}_t + \lambda (U_t - W_t) \\
rJ_t &= p - w_t + \dot{J}_t + \lambda (V_t - J_t) \\
rV_t &= -pc + \dot{V}_t + q (\theta_t) (J_t - V_t) 
\end{align*}
so in our case we have
\begin{equation}
rU_t = z + \dot{U}_t + \theta^{\frac{1}{2}} (W_t - U_t)
\end{equation}
\begin{equation}
rW_t = w_t + \dot{W}_t + \lambda (U_t - W_t)
\end{equation}
\begin{equation}
rJ_t = p - \tau - w_t + \dot{J}_t + \lambda (V_t - J_t)
\end{equation}
\begin{equation}
rV_t = -(p - \tau)c + \dot{V}_t + \theta^{-\frac{1}{2}} (J_t - V_t)
\end{equation}

In the steady state, we can drop the $t$ subscripts. The Nash Bargaining solution for the wage is given by
\[
w = \arg \max (W - U)^{\beta} (J - V)^{1-\beta}
\]
So we have the first order condition (from the product rule)
\[
\beta (W - U)^{\beta - 1} (J - V)^{1-\beta} \frac{d (W-U)}{dw} + (1-\beta) (W - U)^{\beta}  (J - V)^{-\beta} \frac{d (J-V)}{dw} = 0
\]

The expected returns from search $U, V$ are independent of any particular $w$, so we have
\[
\frac{dU}{dw} = \frac{dV}{dw} = 0
\]
and transferable utility implies
\[
\frac{dW}{dw} = -\frac{dJ}{dw} 
\]
so we get that
\[
W - U = \beta ((J-V) + (W-U))
\]
meaning that workers get a share $\beta$ of the total surplus.

Now all that is left from the first order condition is
\[
\beta (J-V) = (1-\beta) (W - U)
\]
Differentiate both sides with respect to $t$ to get
\[
\beta (\dot{J}-\dot{V}) = (1-\beta) (\dot{W} - \dot{U})
\]
Now substitute from the Bellman equations
\begin{align*}
\beta(rJ - p + \tau + w - \lambda (V-J) - (rV + (p - \tau)c - \theta^{-\frac{1}{2}} (J - V))) = \\(1-\beta) (rW - w - \lambda(U-W) - (rU - z - \theta^{\frac{1}{2}} (W - U)))
\end{align*}
Since $\beta r (J-V) = (1-\beta) r (W-U)$, we can simplify to get
\begin{align*}
\beta( - p + \tau + w - \lambda (V-J) - ((p - \tau)c - \theta^{-\frac{1}{2}} (J - V))) = \\(1-\beta) ( - w - \lambda(U-W) - ( - z - \theta^{\frac{1}{2}} (W - U)))
\end{align*}
Since $\beta \lambda (V-J) = (1-\beta) \lambda (U-W)$, we can simplify again to get
\begin{align*}
\beta( - p + \tau + w  - ((p - \tau)c - \theta^{-\frac{1}{2}} (J - V))) = (1-\beta) ( - w  - ( - z - \theta^{\frac{1}{2}} (W - U)))
\end{align*}
Using the same substitution with $\theta^{-\frac{1}{2}}$, we have
\begin{equation}
w = \beta (p - \tau + (p - \tau)c) + (1-\beta) z + \beta (\theta - 1) \theta^{-\frac{1}{2}} (J-V)
\end{equation}

An increase in $\tau$ makes both vacancies and employment less valuable for the firm, so they will not be willing to pay as much for workers.


\subsection{Free entry}

In equilibrium all profit opportunities from new jobs are exploited, driving rents from vacant jobs to zero. Therefore the equilibrium condition for the supply of vacant jobs is $V = 0$. From the Bellman equation for vacancies we get
\[
J = \frac{(p-\tau)c}{\theta^{-\frac{1}{2}}}
\]
and so we will have
\begin{equation}
w = z + \beta(p-\tau-z) + \beta \theta (p - \tau) c
\end{equation}
implying that wage is equal to the sum of the outside option, the bargaining share of the net return to market activity, and the average hiring cost of unemployed workers.

\subsection{Balanced budget}

We need the condition
\[
zu = (1-u) \tau
\]
to hold. In equilibrium, we have the steady state value of $u$ as
\[
u = \frac{\lambda}{\lambda + \theta q (\theta)} =  \frac{\lambda}{\lambda + \theta^{\frac{1}{2}}}
\]
Solving for $\tau$, we have
\begin{equation}
\tau = \frac{z \lambda}{\theta^{\frac{1}{2}}}
\end{equation}
implying a wage of 
\begin{equation}
w = z + \beta(p- z \lambda \theta^{-\frac{1}{2}}-z) + \beta \theta (p -  z \lambda \theta^{-\frac{1}{2}}) c
\end{equation}
where tighter labor markets imply that the cost of jobs is lower for the firms implying higher worker value and therefore higher wages.

\subsection{Equilibrium tightness}

In the steady state we can use the Bellman equation for a job to obtain
\[
J = \frac{p - \tau - w}{r + \lambda}
\]
which we can combine with our free entry condition. 
So we have
\[
\frac{(p-z\lambda \theta^{-\frac{1}{2}})c}{\theta^{-\frac{1}{2}}} = \frac{p - z\lambda \theta^{-\frac{1}{2}} - z + \beta(p- z \lambda \theta^{-\frac{1}{2}}-z) + \beta \theta (p -  z \lambda \theta^{-\frac{1}{2}}) c}{r + \lambda}
\]
Cross multiply, distribute, and divide by $p$ to get
\[
(1-\beta)(1- \frac{z}{p}) - \beta \theta c + \lambda  \frac{\beta \theta + \beta - 1}{\theta^{\frac{1}{2}}} \cdot \frac{z}{p} = \frac{(r+\lambda)c}{\theta^{-\frac{1}{2}}} - \frac{(r+\lambda) \lambda c}{\theta^{\frac{3}{4}}} \cdot \frac{z}{p}
\]
which is the implicit equation desired.

\subsection{Generosity of unemployment benefits}

Solving for the ratio $\frac{z}{p}$ gives
\[
\frac{z}{p} = \frac{ \beta + \beta \theta c + (r+\lambda)c \theta^{-\frac{1}{2}} - 1}{\beta + (r+\lambda) \lambda c + \lambda  (\beta \theta + \beta - 1) \theta^{\frac{1}{2}} -1 }
\]
implying that higher unemployment benefits increase unemployment, as they enhance the outside value of not working and reduce the payoff to jobs.

\subsection{Stability}

The equilibrium is plausibly unstable, as more vacancies increase the value of workers and therefore employment, which may spiral until many more workers are employed.

\section{Unemployment and Business Cycles}

\subsection{Equilibrium tightness}

Apply the job creation and wage determination formulas.
We have
\[
w = z + \beta (p-z) + \beta c \theta
\]
and (from the Bellman equation for a vacancy and free entry)
\[
\frac{c}{q(\theta)} = \frac{p - w}{r + \lambda}
\]
so we can write 
\[
\frac{c}{q(\theta)} = \frac{p - (z + \beta (p-z) + \beta c \theta)}{r + \lambda}
\]
equivalent to 
\[
\frac{r + \lambda}{q(\theta)} = \frac{p - (z + \beta (p-z) + \beta c \theta)}{c}
\]
and
\[
\frac{r + \lambda}{q(\theta)} = \frac{(1-\beta)(p-z) + \beta c \theta)}{c}
\]
and so we will pin down the unique equilibrium with
\[
\beta \theta^* = \frac{(1-\beta)(p-z)}{c} - \frac{r + \lambda}{q(\theta^*)}
\]
and this is unique because the left hand side is increasing in $\theta$ while the right hand side id decreasing in $\theta$ by the properties of the job filling rate. 

\subsection{Elasticity with respect to productivity}

Differentiate with respect to $p$ and multiply by $\frac{p}{\theta^*}$ and we get
\[
\beta \varepsilon_{\theta^*, p} = \frac{p}{\theta^*} \cdot \left(\frac{(1-\beta)}{c} - \frac{(r + \lambda) \frac{dq(\theta)}{dp}}{q(\theta^*)^2} \right)
\]
\[
\beta \varepsilon_{\theta^*, p} = \frac{p}{\theta^*} \cdot \left(\frac{(1-\beta)}{c} - \frac{(r + \lambda) \frac{dq(\theta)}{d\theta} \frac{d\theta}{dp}}{q(\theta^*)^2} \right)
\]
\[
\beta \varepsilon_{\theta^*, p} = \frac{p}{\theta^*} \cdot \left(\frac{(1-\beta)}{c} - \frac{(r + \lambda) \frac{dq(\theta)}{d\theta} \varepsilon_{\theta^*, p} \frac{\theta^*}{p} }{q(\theta^*)^2} \right)
\]
\[
\beta \varepsilon_{\theta^*, p} = \frac{p}{\theta^*} \cdot \left(\frac{(1-\beta)}{c} - \frac{(r + \lambda) \frac{dq(\theta)}{d\theta} \varepsilon_{\theta^*, p} \frac{\theta^*}{pq(\theta)} }{q(\theta^*)} \right)
\]
\[
\beta \varepsilon_{\theta^*, p} = \frac{p}{\theta^*} \cdot \left(\frac{(1-\beta)}{c} + \frac{(r + \lambda) \eta(\theta^*) \varepsilon_{\theta^*, p} \frac{1}{p} }{q(\theta^*)} \right)
\]
and solving for the elasticity gives
\[
\varepsilon_{\theta^*, p} = \frac{1-\beta}{\beta + \frac{r+\lambda}{\theta^* q(\theta^*)} \eta(\theta^*)} \cdot \frac{p}{c\theta^*}
\]
the desired formula.

\subsection{Elasticity with respect to productivity: approximation}

With $\eta \approx 0.5$, small values for $r$ and $\lambda$, along with $\beta \approx \eta$, we have that $\beta \approx 1 - \beta$ and we can pin down market tightness with just
\[
\theta \approx \frac{p-z}{c}
\]
and approximate the elasticity of tightness with respect to aggregate labor productivity
\[
\varepsilon_{\theta^*, p} \approx \frac{p}{c\theta^*}
\]
and so we get the final approximation
\[
\varepsilon_{\theta^*, p} \approx \frac{p}{p-z}
\]
where the right-hand side is the ratio of worker output to the difference between worker output and the outside option.

\subsection{Cost of job loss}

This estimate implies $p/z$ can be approximately 1.05, implying a fairly small cost of job loss/unemployment for workers (since their output is worth only about 5 percent more than the outside option).

\end{document}