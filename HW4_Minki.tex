\documentclass[12pt]{amsart}

\usepackage[dvipsnames]{xcolor}
\usepackage{etoolbox}
\patchcmd{\section}{\normalfont}{\normalfont\color{BrickRed}}{}{}
\patchcmd{\subsection}{\normalfont}{\normalfont\color{blue}}{}{}
\usepackage{lipsum}

\usepackage{geometry}                % See geometry.pdf to learn the layout options. There are lots.
\geometry{a4paper}                   % ... or a4paper or a5paper or ... 
%\geometry{landscape}                % Activate for for rotated page geometry
\usepackage[parfill]{parskip}    % Activate to begin paragraphs with an empty line rather than an indent
\usepackage{enumitem}
\usepackage{tikz}
\usepackage{graphicx}
\usepackage{amssymb}
\usepackage{sgame}
\usepackage{amsmath}
\usepackage{epstopdf}
\DeclareGraphicsRule{.tif}{png}{.png}{`convert #1 `dirname #1`/`basename #1 .tif`.png}

% For Matlab codes
\usepackage{listings}
\usepackage{color} %red, green, blue, yellow, cyan, magenta, black, white
\definecolor{mygreen}{RGB}{28,172,0} % color values Red, Green, Blue
\definecolor{mylilas}{RGB}{170,55,241}

\usetikzlibrary{calc}

\title{Econ 220B Macroeconomics B Problem Set \# 4}
\author{Minki Kim}
%\date{}                                           % Activate to display a given date or no date

\begin{document}
	
% Also for Matlab code
\lstset{language=Matlab,%
	%basicstyle=\color{red},
	breaklines=true,%
	morekeywords={matlab2tikz},
	keywordstyle=\color{blue},%
	morekeywords=[2]{1}, keywordstyle=[2]{\color{black}},
	identifierstyle=\color{black},%
	stringstyle=\color{mylilas},
	commentstyle=\color{mygreen},%
	showstringspaces=false,%without this there will be a symbol in the places where there is a space
	numbers=left,%
	numberstyle={\tiny \color{black}},% size of the numbers
	numbersep=9pt, % this defines how far the numbers are from the text
	emph=[1]{for,end,break},emphstyle=[1]\color{red}, %some words to emphasise
	frame = single
	%emph=[2]{word1,word2}, emphstyle=[2]{style},    
}
	
\maketitle
\section{Consumer's optimization problem}
Consumer/worker's intertemporal constrained optimization problem is written as: 
\begin{align*}
\max &\sum_{t = 0}^{\infty} \beta^t \left( \frac{c_t^{1-\theta}}{1-\theta}  - \gamma \frac{\varepsilon}{1+\varepsilon} h_t^{\frac{1+\varepsilon}{\varepsilon}} \right) \\
\text{s.t. } & c_t + b_{t+1} \leq w_t(s_t) h_t + (1-\delta) b_{t} + r_t(s_t) b_{t} \\
& b_t \geq 0
\end{align*}
A consumer's disposable income consists of labor income, rate of return from the bond purchased in the previous period, and undepreciated part of the bond. She purchases consumptiong goods and also a bond. Negative $b_{t}, \forall t$ is interpreted as borrowing. Condition $b_t \geq 0$ implies even an unemployed (or non-working) individual should be able to pay back her debt. 
\section{Optimal consumption and labor supply}
Set up the Lagrangian:
\begin{align*}
\mathcal{L} = \sum_{t=0}^{\infty} \beta^t \left[  \frac{c_t^{1-\theta}}{1-\theta}  - \gamma \frac{\varepsilon}{1+\varepsilon} h_t^{\frac{1+\varepsilon}{\varepsilon}} + \lambda_t(s_t) \left( w_t(s_t) h_t + (1-\delta) b_{t} + r_t(s_t) b_{t} - c_t - b_{t+1}  \right) + \phi_t(s_t) b_t \right]
\end{align*}
First order conditions are: 
\begin{align*}
&c_t^{-\theta} = \lambda_t(s_t) \\
& \gamma h_t^{\frac{1}{\varepsilon}} = \lambda_t(s_t) w_t(s_t) \\
& \lambda_t(s_t) = \beta \lambda_{t+1}(s_{t+1}) \left( 1- \delta + r_t(s_t)  \right) + \phi_t(s_t)
\end{align*}
Combining the first and third conditions, we get the Euler conditions 
\begin{align*}
\begin{cases}
c_t^{-\theta} = \beta c_{t+1}^{-\theta} \left( 1- \delta + r_{t+1}(s_{t+1})  \right) & \text{Case (1) when } b_t > 0 \\    
c_t^{-\theta} = \beta c_{t+1}^{-\theta} \left( 1- \delta + r_{t+1}(s_{t+1})  \right)  +  \beta \phi_{t+1}(s_{t+1}) & \text{Case (2) when } b_t = 0    
\end{cases}
\end{align*}
When the borrowing constraint is not binding(Case (1)), we have usual Euler equation. When borrowing constraint is binding, $\phi_{t+1}(s_{t+1})$, the Lagrange multiplier attached to the borrowing constraint, shows up in the Euler equation. Binding borrowing constraint means that the consumer is borrowing up to his limit to increase current period consumption as much as he can. Therefore, LHS denotes the marginal gain of increased consumption. The first term of RHS denotes the discounted loss in period $t+1$, coming from increased consumption in period $t$. The second term entails discounted marginal loss coming from forgone bond holdings. \\

For labor supply elasticity, note that $h_t = \left( \frac{1}{\gamma} c_t^{-\theta} w_t \right)^\varepsilon$. It is easy to show that $\epsilon_{h,w} = \frac{\partial h_t}{\partial w_t} \frac{w_t}{h_t} = \varepsilon$. 
\section{Firm's labor demand}
Firm's static maxization problem is written as: 
\begin{equation*}
\max_{k_t, n_t} A_t(s_t) k_t^\alpha n_t^{1-\alpha} - w_t(s_t) n_t - r_t(s_t) k_t
\end{equation*}
From the first order condition for labor we get $w_t(s_t) = (1-\alpha) \frac{y_t(s_t)}{n_t}$
\section{Equilibrium employment level}
Labor market is cleared when $h_t = n_t$, i.e. 
\begin{align*}
&h_t^{*} = \left( \frac{1}{\gamma} c_t^{-\theta} w_t \right)^\varepsilon = \left( \frac{1}{\gamma} c_t^{-\theta} (1-\alpha) \frac{y_t}{h_t^{*}} \right)^\varepsilon \\
&\text{by isolating $h_t^{*}$, we get} \\
&h_t^{*} = \left(  \frac{1}{\gamma} c_t^{-\theta} (1-\alpha) y_t  \right)^{\frac{\varepsilon}{1+\varepsilon}}
\end{align*}

\section{Calibrating labor supply elasticity}
Take log to $h_t = \left(  \frac{1}{\gamma} c_t^{-\theta} (1-\alpha) y_t  \right)^{\frac{\varepsilon}{1+\varepsilon}}$ and calculate its variance. 
\begin{align*}
Var(\log h_t) &= Var \left[ \frac{\varepsilon}{1+\varepsilon} \left( -\log \gamma -\theta \log c_t + \log (1-\alpha) + \log y_t  \right) \right] \\
& = \left( \frac{\varepsilon}{1+ \varepsilon}\right)^2 \left[ Var(\log y_t) + \theta^2 Var(\log c_t) - 2\theta Cov(\log y_t, \log c_t) \right] \\
& = \left( \frac{\varepsilon}{1+ \varepsilon}\right)^2 \left[ Var(\log y_t) +  Var(\log c_t) - 2 Cov(\log y_t, \log c_t) \right] (\because \theta = 1) \\
& = \left( \frac{\varepsilon}{1+ \varepsilon}\right)^2 \Bigg[ Var(\log y_t) +  Var(\log c_t) \\
& \hspace{3cm} - 2 \left( \sqrt{Var(\log y_t)} \sqrt{Var(\log c_t)} \right)\rho(\log y_t, \log c_t) \Bigg]
\end{align*}
Normalize $Var(\log y_t) = 1$, Then $Var(\log c_t) = 9/16, Var(\log h_t) = 9/25$. Also, $\rho(\log y_t, \log c_t)$ is given as $3/4$. We get $\varepsilon = 9.7660$ under these conditions. Hence the labor elasticity implied by the model by far exceeds the upper bound of microeconometric estimates. 
\section{Comparative statics: $\theta = 2$}
\begin{align*}
Var(\log h_t) & = \left( \frac{\varepsilon}{1+ \varepsilon}\right)^2 \left[ Var(\log y_t) +  4 Var(\log c_t) - 4 Cov(\log y_t, \log c_t) \right] \\
& = \left( \frac{\varepsilon}{1+ \varepsilon}\right)^2 \Bigg[ Var(\log y_t) + 4 Var(\log c_t) \\
& \hspace{3cm} - 4 \left( \sqrt{Var(\log y_t)} \sqrt{Var(\log c_t)} \right)\rho(\log y_t, \log c_t) \Bigg]
\end{align*}
We get $\varepsilon = 1.5$ under the alternative calibration. Suppose that the consumer was temporarily hit by a positive wage income shock. Since higher $\theta$ reduces the intertemporal elasticity of substitution (or induces a stronger motive for consumption smoothing), the consumer increases his consumption by less amount than he would do under $\theta=1$.  Because increased consumption is supported by larger labor income from working longer, less $\Delta c_t$ is translated into less $\Delta h_t$, which drives down the required value of $\varepsilon$. 
\section{Indivisible Labor}
Re-write the consumer's sequential problem when whether she supplies labor is determined by lotteries. 
\begin{align*}
\max &\sum_{t = 0}^{\infty} \beta^t \left( \frac{c_t^{1-\theta}}{1-\theta}  - \gamma \frac{\varepsilon}{1+\varepsilon} h_t^{\frac{1+\varepsilon}{\varepsilon}} \right) \\
\text{s.t. } & c_t + b_{t+1} \leq w_t(s_t) h_t + (1-\delta) b_{t} + r_t(s_t) b_{t} \\
& b_t \geq 0
\end{align*}
\end{document}  