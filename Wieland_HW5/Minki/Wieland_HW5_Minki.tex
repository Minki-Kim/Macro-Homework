\documentclass[11pt]{amsart}
\usepackage{geometry}                % See geometry.pdf to learn the layout options. There are lots.
\geometry{a4paper}                   % ... or a4paper or a5paper or ...
%\geometry{landscape}                % Activate for for rotated page geometry
\usepackage[parfill]{parskip}    % Activate to begin paragraphs with an empty line rather than an indent
\usepackage{enumitem}
\usepackage{graphicx}
\usepackage{amssymb}
\usepackage{amsmath}
\usepackage{cancel}
\usepackage{epstopdf}
\DeclareGraphicsRule{.tif}{png}{.png}{`convert #1 `dirname #1`/`basename #1 .tif`.png}
\usepackage{breqn}
\usepackage{float}
\usepackage{breqn}

\title{Econ 210C Problem Set \# 5}
\author{Minki Kim}
%\date{}                                           % Activate to display a given date or no date

\begin{document}




\maketitle

\section{Problems from Romer}
\subsection{Romer, Problem 6.13.}
\subsection{Romer, Problem 7.10.}
\section{Quadratic cost of adjusting prices and effect of money \\ (Rotemberg 1982)}
\begin{enumerate}[label = (\alph*)]
	\item 
	\item 
	\item 
	\item 
	\item
\end{enumerate}
\section{New Keynesian model in Dynare}
\begin{enumerate}[label = (\alph*)]
	\item 
	\item 
	\item 
	\item 
\end{enumerate}
\section{Government spending multipliers in the new Keynesian model (Christiano, Eichenbaum and Evans 2012)}
\begin{enumerate}[label = (\alph*)]
	\item The economy is characterized by the following log-linearized equations:
	\begin{align*}
	 \tilde { C } _ { t } &= E _ { t } \tilde { C } _ { t + 1} - \frac { 1} { \psi } \left( i _ { t } - E _ { t } \pi _ { t + 1} \right)  \\ 
	 \pi _ { t } &= \beta E _ { t } \pi _ { t + 1} + \kappa \left( \frac { \tilde {W} } { P } \right)_t ,\quad \kappa = \frac { ( 1- \theta ) ( 1- \beta \theta ) } { \theta }  \\
	\left( \frac { \tilde{ W } } { P } \right) _ { t } &= \psi \tilde { C } _ { t } + \frac { 1} { \eta } L _ { t } \\
	\tilde { Y } _ { t } &= \check { L } _ { t } \\
	\tilde{ Y }_t &= s_g \tilde{G}_t + (1-s_g) \tilde{ C }_t \\
	i_t &= \phi_\pi \pi_t, \quad \phi_\pi > 1
	\end{align*}
	The first equation is a standard Euler equation. The second equation is a recursive formulation of inflation rate, telling us that current inflation is a present value of future marginal costs. The third equation is household's labor supply. The fourth equation denotes aggregate production function. The fifth equation is national account, where $s_g$ is the share the government spending. Finally, the last equtaion implies that the central bank follows the Taylor rule. 
	\item 
	\item 
	\item 
	\item
	\item 
	\item 
	\item 
	\item 
	\item 
	\item 
	\item 
\end{enumerate}

\end{document}
