\documentclass[11pt]{amsart}
\usepackage{geometry}                % See geometry.pdf to learn the layout options. There are lots.
\geometry{a4paper}                   % ... or a4paper or a5paper or ...
%\geometry{landscape}                % Activate for for rotated page geometry
\usepackage[parfill]{parskip}    % Activate to begin paragraphs with an empty line rather than an indent
\usepackage{enumitem}
\usepackage{graphicx}
\usepackage{amssymb}
\usepackage{amsmath}
\usepackage{cancel}
\usepackage{epstopdf}
\DeclareGraphicsRule{.tif}{png}{.png}{`convert #1 `dirname #1`/`basename #1 .tif`.png}
\usepackage{breqn}
\usepackage{float}
\usepackage{breqn}

\title{Econ 210C Problem Set \# 5}
\author{Minki Kim}
%\date{}                                           % Activate to display a given date or no date

\begin{document}




\maketitle

\section{Problems from Romer}
\subsection{Romer, Problem 6.13.}
\begin{enumerate}[label = (\alph*)]
	\item 
	\item 
	\item 
	\item 
	\item 
	\item
\end{enumerate}
\subsection{Romer, Problem 7.10.}
\section{Quadratic cost of adjusting prices and effect of money \\ (Rotemberg 1982)}
\begin{enumerate}[label = (\alph*)]
	\item 
	\item 
	\item 
	\item 
	\item
\end{enumerate}
\section{New Keynesian model in Dynare}
\begin{enumerate}[label = (\alph*)]
	\item 
	\item 
	\item 
	\item 
\end{enumerate}
\section{Government spending multipliers in the new Keynesian model (Christiano, Eichenbaum and Evans 2012)}
\begin{enumerate}[label = (\alph*)]
	\item The economy is characterized by the following log-linearized equations:
	\begin{align*}
	 \check { C } _ { t } &= E _ { t } \check { C } _ { t + 1} - \frac { 1} { \psi } \left( i _ { t } - E _ { t } \pi _ { t + 1} \right)  \\ 
	 \pi _ { t } &= \beta E _ { t } \pi _ { t + 1} + \kappa \left( \frac { \check {W} } { P } \right)_t ,\quad \kappa = \frac { ( 1- \theta ) ( 1- \beta \theta ) } { \theta }  \\
	\left( \frac { \check{ W } } { P } \right) _ { t } &= \psi \check { C } _ { t } + \frac { 1} { \eta } L _ { t } \\
	\check { Y } _ { t } &= \check { L } _ { t } \\
	\check{ Y }_t &= s_g \check{G}_t + (1-s_g) \check{ C }_t \\
	i_t &= \phi_\pi \pi_t, \quad \phi_\pi > 1
	\end{align*}
	The first equation is a standard Euler equation. The second equation is a recursive formulation of inflation rate, telling us that current inflation is a present value of future marginal costs. The third equation is household's labor supply. The fourth equation denotes aggregate production function. The fifth equation is national account, where $s_g$ is the share the government spending. Finally, the last equtaion implies that the central bank follows the Taylor rule. 
	
	\item The reduced system is characterized as follows: 
	\begin{align*}
	\check{ C }_t &= E_t \check{ C }_{t+1} - \frac{1}{\psi} \left(\phi_\pi \pi_t - E_t\pi_{t+1} \right) \\
	\pi_t &= \beta E_t \pi_{t+1} + \kappa \left( \psi \check{ C }_t + \frac{s_g}{\eta} \check{ G}_t + \frac{(1-s_g)}{\eta} \check{ C }_t \right)
	\end{align*}
	We have two endogenous variables ($\check{ C }_t, \pi_t$) and one exogenous variable ($\check{ G}_t$). 
	
	\item Assume that government spending has the following dynamics: 
	\begin{equation*}
	\check { G } _ { t } = \rho \check { G } _ { t - 1} + \epsilon _ { t } ,\quad \epsilon _ { t } \sim i .i .d .\left( 0,\sigma ^ { 2} \right)
	\end{equation*} 
	Since both $\check{ C }_t$ and $\pi_t$ are jump-variables, the only state variable is $\check{ G}_t$.  $1$ is not a state variable because the model is log-linearized. Since $\check{ C }_{t-1}$ and $\pi_{t-1}$ do not show up in the system, lagged endogenous variables are also not state variables. 
	
	\item Since rational expectation was introduced to the literature, numerous researchers have commented on the multiplicity of solution paths in linear rational expectation models. Since models with inifinite number of solutions which are consistent with rational expectation are evidently unusable, seeking for a solution procedure which can single out a unique, bubble-free solution was a natural flow of the literature. 
	
	McCallum(1983) suggests a solution procedure called the \textit{minimum state variable (MSV) criterion}. Since the multiplicity (or sunspots) arises from redundant state variables which are unnecessary but not formally inconsistent with rational expectations, he suggests to rule out those state variable from the beginning. In our model, the minimum set of state variables is $\{\check{ G }_t\}$. The solution equations for $\check{ C }_t$ and $\pi_t$ are expressed as functions of state variable $\check{ G }_t$:
	\begin{align*}
	\check{ C }_t &= c_g \check{ G}_t \\
	E_t \check{ C }_{t+1} &=c_g \rho \check{ G}_t \\
	\pi_t &= \pi_g \check{ G }_t \\
    E_t \pi_{t+1} & = \pi_g \rho \check{ G }_t	
	\end{align*}
	Plugging in these guessed form of solution equations into the system: 
	\begin{gather*}
	c_g \check{ G}_t = c_g \rho \check{ G}_t - \frac{1}{\psi} \left(\phi_\pi \pi_g \check{ G }_t - \pi_g \rho \check{ G }_t\right) \\
	\pi_g \check{ G }_t = \beta \pi_g \rho \check{ G }_t + \kappa \left( \psi c_g \check{ G}_t + \frac{s_g}{\eta} \check{ G}_t + \frac{(1-s_g)}{\eta} c_g \check{ G}_t \right)
	\end{gather*}
	With some algebraic effort, we get 
	\begin{align*}
	\pi _ { g } &= \frac { \frac { \kappa } { \eta } ( 1- \rho ) s _ { g } } { ( 1- \beta \rho ) ( 1- \rho ) + \frac { \kappa } { \psi } \left[ \psi + \frac { 1} { \eta } \left( 1- s _ { g } \right) \right] \left( \phi _ { \pi } - \rho \right) } \\
	c _ { g } &= \frac { - \frac { \kappa } { \eta \psi } \left( \phi _ { \pi } - \rho \right) s _ { g } } { \left( 1 -  \beta \rho \right) ( 1- \rho ) + \frac { \kappa } { \psi } \left[ \psi + \frac { 1} { \eta } \left( 1- s _ { g } \right) \right] \left( \phi _ { \pi } - \rho \right) }
	\end{align*}
	\item Following the solutioin obtained in (d), 
	\begin{align*}
	\frac{d Y}{d G} =  1 + \frac{d C}{d G} &= 1 + \frac{1-s_g}{s_g} \frac{d \check{ C }}{d \check{ G }} \\
	& = 1 + \frac{c_g (1-s_g)}{s_g} \\
	& = 1 - \frac {  \frac { \kappa } { \eta \psi } \left( \phi _ { \pi } - \rho \right) (1-s _ { g }) } { \left( 1 -  \beta \rho \right) ( 1- \rho ) + \kappa \left[ 1 + \frac { 1} { \eta \psi } \left( 1- s _ { g } \right) \right] \left( \phi _ { \pi } - \rho \right) } \\
	& = 1 - \frac {  \frac { \kappa } { \eta \psi } \left( \phi _ { \pi } - \rho \right) (1-s _ { g }) } { \left( 1 -  \beta \rho \right) ( 1- \rho ) + \kappa(\phi_\pi - \rho) + \frac { \kappa } { \eta \psi } \left( 1- s _ { g } \right)  \left( \phi _ { \pi } - \rho \right) } \\
	& \qquad \text{Let } \frac { \kappa } { \eta \psi } \left( \phi _ { \pi } - \rho \right) (1-s _ { g }) = A \\
	& = 1 - \frac{A}{\left( 1 -  \beta \rho \right) ( 1- \rho ) + \kappa(\phi_\pi - \rho) + A} 
	\end{align*}
	\item Since $\left( 1 -  \beta \rho \right) ( 1- \rho ) + \kappa(\phi_\pi - \rho) >1$, $	dY / dG  < 1$. The government spending multiplier is smaller than one because the Taylor rule effectively stabilizes the inflation, hence cools down the economy.
	\item When $\phi = 0$, the government spending multiplier is:
	\begin{align*}
	\frac{d Y}{d G} &=  1 + \frac{\frac{\kappa}{\eta \phi} \rho(1-s_g)}{(1-\beta \rho)(1-\rho) - \frac{\kappa}{\psi} \left[ \psi + \frac{1}{\eta} (1-s_g) \right] \rho } \\
	& > 1
	\end{align*}
	So the government spending multiplier is larger than 1 even for a very low persistence shock. This is because the monetary policy is not raising nominal interest rate enough to make the real interest rate $r_t = i_t - E_t \pi_{t+1}$ also rise. Since the real interest rate is too low, even a small demand shock can make a ripple effect. 
	
	\item Assuming $\phi_\pi = 0$, or generally any value of $\phi_\pi$ which violates the Taylor principle ($\phi_\pi >1$) is problematic because the model becomes explosive. 
	\item The economy in the ZLB is characterized by the following system of equations: 
	\begin{align*}
	\check { C } _ { t } &= E _ { t } \check { C } _ { t + 1} - \frac { 1} { \psi } \left( i_t - E _ { t } \pi _ { t + 1} \right)  \\ 
	\pi _ { t } &= \beta E _ { t } \pi _ { t + 1} + \kappa \left( \frac { \check {W} } { P } \right)_t ,\quad \kappa = \frac { ( 1- \theta ) ( 1- \beta \theta ) } { \theta }  \\
	\left( \frac { \check{ W } } { P } \right) _ { t } &= \psi \check { C } _ { t } + \frac { 1} { \eta } L _ { t } \\
	\check { Y } _ { t } &= \check { L } _ { t } \\
	\check{ Y }_t &= s_g \tilde{g} + (1-s_g) \check{ C }_t \\
	i_t &= -\bar{i}
	\end{align*}
	The reduced 2-equation system is: 
	\begin{align*}
	\check { C } _ { t } &= E _ { t } \check { C } _ { t + 1} - \frac { 1} { \psi } \left( -\bar{i} - E _ { t } \pi _ { t + 1} \right) \\
	\pi_t &= \beta E_t \pi_{t+1} + \kappa \left( \psi \check{ C }_t + \frac{s_g}{\eta} \tilde{g} + \frac{(1-s_g)}{\eta} \check{ C }_t \right)
	\end{align*}
	Substituting guessed forms of solutions (without constant terms) into the system yields: 
	\begin{align*}
	c_g^{ZLB}\tilde{g} &= p c_g^{ZLB} \tilde{g} - \frac { 1} { \psi } \left( -\bar{i} - p \pi_g^{ZLB} \tilde{g} \right) \\
	\pi_g^{ZLB} \tilde{g} &= \beta p \pi_g^{ZLB} \tilde{g} + \kappa \left( \psi c_g^{ZLB} \tilde{g} + \frac{s_g}{\eta} \tilde{g} + \frac{(1-s_g)}{\eta} c_g^{ZLB}\tilde{g} \right)
	\end{align*}
	Solving the system gives us 
	\begin{align*}
	\pi _ { g } &= \frac { \frac { \kappa } { \eta } ( 1- p ) s _ { g } } { ( 1- \beta p ) ( 1- p ) - \frac { \kappa } { \psi } \left[ \psi + \frac { 1} { \eta } \left( 1- s _ { g } \right) \right] p } \\
	c _ { g } &= \frac { \frac { \kappa } { \eta \psi } p s _ { g } } { ( 1- \beta p ) ( 1- p ) - \frac { \kappa } { \psi } \left[ \psi + \frac { 1} { \eta } \left( 1- s _ { g } \right) \right] p }
	\end{align*}
	\item Following the solutioin obtained in (i), 
	\begin{align*}
	\frac{d Y}{d G} =  1 + \frac{d C}{d G} &= 1 + \frac{1-s_g}{s_g} \frac{d \check{ C }}{d \check{ G }} \\
	& = 1 + \frac{c_g^{ZLB} (1-s_g)}{s_g} \\
	& = 1 + \frac { \frac { \kappa } { \eta \psi } p (1-s_g) } { ( 1- \beta p ) ( 1- p ) - \frac { \kappa } { \psi } \left[ \psi + \frac { 1} { \eta } \left( 1- s _ { g } \right) \right] p }
	\end{align*}
	\item Given $( 1- \beta p ) ( 1- p ) - \frac { \kappa } { \psi } \left[ \psi + \frac { 1} { \eta } \left( 1- s _ { g } \right) \right] p > 0$, the government spending multiplier is always larger than 1. This is because the Taylor rule is not working at the ZLB, so the monetary policy does not stabilize the economy. 
	\item Compare the two government spending multipliers
	\begin{align*}
	\text{in (g): }& \quad 	\frac{d Y}{d G} =  1 + \frac{\frac{\kappa}{\eta \phi} \rho(1-s_g)}{(1-\beta \rho)(1-\rho) - \frac{\kappa}{\psi} \left[ \psi + \frac{1}{\eta} (1-s_g) \right] \rho }\\
	\text{in (k): }& \quad \frac{d Y}{d G} = 1 + \frac { \frac { \kappa } { \eta \psi } p (1-s_g) } { ( 1- \beta p ) ( 1- p ) - \frac { \kappa } { \psi } \left[ \psi + \frac { 1} { \eta } \left( 1- s _ { g } \right) \right] p }
	\end{align*}
	Those two multipliers are the same when $p = \rho$. In (g), monetary policy is ineffective indefinitely. Thus the persistence of the government spending shock determines the size of the multiplier. In (k), on the other hand, the probability of escaping form the ZLB determines the size of the multiplier, since $\rho=1$ as long as the economy is trapped in the ZLB. 
\end{enumerate}

\end{document}
